\label{sec:chap2-single-rl}
In this section, we will go through the classical single agent reinforcement learning as a basis to the next section, which is about deep reinforcement learning - a scale and approximated version of algorithms discuss here. The contents are re-arranged from \cite{sutton2018reinforcement} and \cite{silver}.

\subsection{Markov Decision Process}
\label{sec:chap2-rl-defintions}
Markov Decision Process (MDP) formulates the task that reinforcement learning trying to solve\footnote{It is a mathematical description of \correctquote{game}.}. 
\begin{definition}
    Markov Decision Process (MDP) is a tuple : $\langle S, A, R, T, p_0, \gamma \rangle$ where 
    \begin{itemize}
        \item State space: $S = \{s_0, s_1, \cdots , s_{|S|}\}$. Usually we have $s_i \in \mathbb{R}^{d_s}$
        \item Action space: $A = \{a_0, a_1, \cdots a_{|A|}\}$
        \item Reward function: $\mathcal{R}: S \times A \rightarrow \mathbb{R}$
        \item Transition function: $T: S \times A \times S \rightarrow [0, 1]$, where $s^{(t+1)} \sim T(s^{(t+1)} | s^{(t)}, a^{(t)})$ 
        \item Initial state distribution: $p_0 : S \rightarrow [0, 1] $, where $s^{(0)} \sim p_0$
        \item Discounted factor: $\gamma \in (0, 1]$
    \end{itemize}
    We define player's policy $\pi: S \rightarrow \Delta(A)$ where $\Delta(A)$ is probability simplex over $A$, and $a^{(t)} \sim \pi(a^{(t)} | s^{(t)})$ denote player choose action $a_t$ at state $s_t$ at time $t$. The goal of any reinforcement learning algorithms is to maximizes the sum reward with or without full description/experience of MDP.
\end{definition}
\noindent
We denote the value function of the player's policy $V^\pi(s)$ as the expected future cumulative reward given the state the agent is in:
\begin{equation}
    V^\pi(s) = \mathbb{E}_{a_t \sim \pi, s_t \sim T}\brackb{\sum^\infty_{t=0} \gamma^t r(s_t, a_t) \Bigg| s_0 = s}
\end{equation}
And, the action-value function is the expected future reward when the agent performs an action $a$ at state $s$: 
\begin{equation}
    Q^\pi(s, a) = \mathbb{E}_{a_t \sim \pi, s_t \sim T}\brackb{\sum^\infty_{t=0} \gamma^t r(s_t, a_t) \Bigg| s_0 = s, a_0 = a}
\end{equation}
The advantage function of an agent is simply the differences between value function and action value function i.e $A^{\pi}(s, a) = Q^\pi(s, a) - V^\pi(s)$. Furthermore, We establish the connection between both $V(s)$ and $Q(s, a)$ functions as
\begin{equation}
    \label{eqn:chap2-Q-and-V}
    Q^\pi(s, a) = r(s, a) + \gamma \mathbb{E}_{s' \sim T(s' | s, a)}\brackb{V^\pi(s')} \quad \quad \quad V^\pi(s) = \mathbb{E}_{a \sim \pi(a | s)} \brackb{Q^\pi(s, a)}
\end{equation}
We denote the optimal value function and optimal action value function as $V^*(s)$ and $Q^*(s, a)$ when $V^{\pi^*}(s)$ and $Q^{\pi^*}(s, a)$, respectively, where $\pi^* \in \arg\max_{\pi} V^{\pi}(s)$ for any state $s$. Now, we want to establish the connection between these 2 optimal functions, which is:
\begin{equation}
\label{eqn:chap2-optim-Q-and-V}
    Q^*(s, a) = r(s, a) + \gamma \mathbb{E}_{s' \sim T(s' | s, a)}\brackb{V^*(s')} \quad \quad \quad V^*(s) =\max_{a\in A} Q^*(s, a)
\end{equation}
We can now slightly change the objective of reinforcement learning to \correctquote{How to find such an optimal policy}.

\subsection{Policy Iteration}
\label{sec:chap2-policy-iter}
We will now introduce the first algorithm that will solve the problem of reinforcement learning, called policy iteration. The algorithm is based on a simple observation that the agent will always be improving or remain the same if we greedily choose the action that would maximize expected future rewards given $Q^\pi$:
\begin{equation}
    \label{eqn:chap2-greedy-Q}
    \pi'(a | s) = \begin{cases}
        1 &\text{ if } a = \argmax{a \in A} Q^\pi(s, a) \\
        0 &\text{ otherwise }
    \end{cases}
\end{equation}
We call this step, a policy improvement step. One can show that $\forall s \in S: V^\pi(s) \le V^{\pi'}(s)$, and we are able to reach an optimal value function if we keep improving the policy. The proof, for completeness, will be presented in appendix \ref{appx:chap2-rl-policy-improve}. What we have left is to find an algorithm that gives us value function $V^{\pi}(s)$ for any policy $\pi$, which we will call this step \textit{policy evaluation}. We can expand the equality in equation \ref{eqn:chap2-Q-and-V} as:
\begin{equation}
    V^\pi(s) = \mathbb{E}_{a \sim \pi(a | s)} \brackb{r(s, a) + \gamma \mathbb{E}_{s' \sim T(s' | s, a)}\brackb{V^\pi(s')}} 
\end{equation}
We call this equation expected Bellman equation along with the following expected Bellman operator $\contractop^{\pi} : \mathbb{R}^{|S|} \rightarrow \mathbb{R}^{|S|}$ defined as:
\begin{equation}
    \label{eqn:chap2-exp-bellman-operator}
    \contractop^{\pi} V(s) = \mathbb{E}_{a \sim \pi(a | s)} \brackb{r(s, a) + \gamma \mathbb{E}_{s' \sim \contractop(s' | s, a)}\brackb{V(s')}} 
\end{equation}
To solve the policy evaluation problem, we want to find $V(s)$ such that $\contractop^\pi V(s) = V(s)$ however, to solve this via matrix inversion would be too expensive. However, turn out this expected Bellman operator is a contraction mapping on $\infty$-norm i.e 
\begin{equation*}
    \| \contractop^\pi V_1(s) - \contractop^\pi V_2(s) \|_\infty \le \alpha \|V_1(s) - V_2(s)\|_\infty
\end{equation*}
where $\|V(s)\|_\infty = \max_{s \in S} V(s)$. For some $0 \le \alpha < 1$ (See appendix \ref{appx:chap2-rl-expected-bell-contract} for proof), then by Banach fixed-point theorem
\begin{theorem}{(Banach fixed-point theorem \cite{murfet_2019})}
    Given the complete (every Cauchy sequence converges to a point in that space) metric space $(X, d(\cdot, \cdot))$ and the contraction mapping $\contractop : X \rightarrow X$, the sequence $(\contractop^{(n)} x)^{\infty}_{n=1}$ converges to a fixed point $x^*$ where $\contractop x^* = x^*$, for all points $x \in X$. 
\end{theorem}
repeatedly apply the Bellman operator will lead us to a solution of $V^{\pi}$.  In conclusion, Policy iteration is an algorithm that solves the MDP by involving the alternation between the following two steps.
\begin{enumerate}
    \item \textbf{Policy Evaluation}: Starting with randomized value function and repeatedly applying Bellman operator defined in equation \ref{eqn:chap2-exp-bellman-operator} until converge to the true value function $V^{\pi}(s)$.
    \item \textbf{Policy Improvement}: Update the policy by choosing the best action from action-value function (we can calculate this by the equation \ref{eqn:chap2-Q-and-V}) following the equation \ref{eqn:chap2-greedy-Q}.
\end{enumerate}
Since the value function always increases in every state, this algorithm is guaranteed to reach the optimal value function and policy. 

\subsection{Value Iteration}
\label{sec:chap2-value-iter}
Now, it is computational ineffective to evaluate the policy every time to update the policy. Can we update the value function using single Bellman update to reach the optimal value function? The answer is \textit{Yes}. We define optimal value Bellman update operator as:
\begin{equation}
    \label{eqn:chap2-optimal-bellman-operator}
    \contractop^*_V V(s) = \max_{a \in A}\Big[r(s, a) + \gamma \mathbb{E}_{s' \sim T(s' | s, a)}\brackb{V(s')} \Big]
\end{equation}
Note that it is related to optimal Bellman equation represented in equation \ref{eqn:chap2-optim-Q-and-V}, where $V^*$ is a fixed point. In this case, we update the value function toward the action that yields the highest value given only one update. This operator is, indeed, a contraction mapping, thus repeatedly updating the value function by optimal value Bellman operator will provide us with optimal value function. The proof will be similar to the proof done in the section above. Finally, given the optimal value function, one can calculate the optimal action-value function and the optimal policy. 

\subsection{Q Iteration}
\label{sec:chap2-Q-iter}
Now, instead of using value function as in value iteration, it is always more desirable for us to estimate optimal action-value function $Q^*(s, a)$ directly. The optimal action-value function has an advantage, in which we can find the best action directly without any extra calculation. One can define the optimal action-value Bellman operator as 
\begin{equation}
    \contractop^*_Q Q(s, a) = r(s, a) + \gamma \mathbb{E}_{s' \sim T(s' | s, a)}\brackb{\max_{a\in A} Q(s', a)} 
\end{equation}
this operator is indeed a contraction mapping with $Q^*$ as its fixed point. The proof is similar compare to other algorithms. However, there is more to this. Since there is no need to consider the transition function explicitly, we defined the following update rule given iteration $k$ and $(s_t, a_t)$ experience:
\begin{equation}
\label{eqn:chap2-Q-stochastic-update}
    Q^{(k+1)}(s_t, a_t) \leftarrow (1-\alpha_t) Q^{(k)}(s_t, a_t) + \alpha_t\brackb{r(s_t, a_t) + \gamma \max_{a'} Q^{(k)}(s_{t+1}, a')}
\end{equation}
where $0 < \alpha < 1$. Following the theorem 1 of \cite{jaakkola1994convergence}, which is also a basis of proofing Q-Iteration states that:
\begin{theorem}
\label{thm:updaate-stochastic}{\cite{jaakkola1994convergence}}
    The process $\Delta_{t+1} = (1-\alpha_t(x)) \Delta_t + \alpha_t(x) F_n(x) $ will converge to zero if and only if The state space is finite, $\sum_t \alpha_t(x) = \infty$, $\sum_t \alpha_t(x)^2 < \infty$, $\left\|\mathbb{E}\brackb{F_n(x) \big| P_n} \right\| \le \beta \|\Delta_n\|$ where $\beta \in (0, 1)$ and  $\operatorname{Var}\brackb{F_n(x) \big| P_n} \le C\bracka{1 + \|\Delta_n\|}^2$, where $P_n$ is a collection of histories. 
\end{theorem}
By subtracting optimal Q function at both ends i.e $\Delta_t = Q^{(k)}(s_t, a_t) - Q^*(s_t, a_t)$ and $F_n(x) = r(s_t, a_t) + \gamma \max_{a'} Q^{(k)}(s_{t+1}, a') - Q^*(s_t, a_t)$, we can proof that the update in equation \label{eqn:chap2-Q-stochastic-update} will converge to the optimal Q-value. Given Theorem \ref{thm:updaate-stochastic}, we can now change any contraction mapping into a stochastic update rule and still maintains the convergence property. We will use this throughout the text.

\subsection{Partially observable Markov decision process}
Before we move to deep reinforcement learning, we shall consider another problem that requires special attention, which will be useful in chapter \ref{chapter:chap5}. Partial observable MDP (POMDP) is defined as 
\begin{definition}
    We define a POMDP as a tuple $\langle S, A, O, \mathcal{T}, p_0, U, R, \gamma\rangle$ where
    \begin{itemize}
        \item State space: $S = \{s_0, s_1, \cdots , s_{|S|}\}$
        \item Action space: $A = \{a_0, a_1, \cdots a_{|A|}\}$
        \item Observation space: $O = \{o_0, o_1, \cdots, o_{|O|}\}$
        \item Transition function: $T: S \times A \times S \rightarrow [0, 1]$
        \item Initial state distribution: $p_0 : S \rightarrow [0, 1]$
        \item Noisy or partially occluded observation: $o_{t+1} \sim U(o_{t+1} | s_{t+1}, a_t)$
        \item Reward distribution function: $r_{t+ 1} \sim R(r_{t+1} | s_{t+1}, a_t)$
        \item Discount factor $\gamma \in \mathbb{R}$ 
    \end{itemize}
    The policy conditions its action based on histories of observation $\pi(a_t | o_{\le t})$ where $o_{\le t}$ is the history of observation before time $t+1$. 
\end{definition}