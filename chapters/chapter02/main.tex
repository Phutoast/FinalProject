\label{chapter:chap2}
\begin{adjustwidth}{1cm}{}
    % In this chapter, we will provide and reviews tools that will be useful for deriving probabilistic multi-agent reinforcement learning algorithms with a formal setting of the problem that the algorithms are trying to solve. We will start by going through the probabilistic machine learning, which provides the tools for mitigate an inevitable intractable problem (will discuss in more details). After getting all the fundamental tools, we will move to single agent reinforcement learning section, which, in the first part, will provide classical results, and theorem. In the later section, we will investigate control-as-inference framework and its extensions. Finally, we will finish with an introduction to multi-agent reinforcement learning with Multi-agent with probabilistic inference framework (MAPI) - our main focus. Our main focus for this chapter is to provide readers backgrounds of the current sub-field, and how they are developed, which will help them understand the motivation for our result more clearly.
    The main aim of this chapter is to make the readers familiar with the background required to understand the development of the algorithms proposed in this thesis. We try to keep this chapter as small as possible. As noted before, any specific algorithms that will only be discussed in a particular chapter will not be described here; this will also help to find related information more accessible. At the end of this chapter, in section \ref{sec:chap2-other-algos}, we will list the algorithms that will be introduced, given the corresponding chapters.
\end{adjustwidth}


\section{Probabilistic Method in Machine Learning}
\label{sec:chap2-prob-ml}

\subsection{Introduction To Baysian Method}
\label{sec:chap2-intro-baysian}
Most of the focus in this section will be to provide techniques that provides an approximate solution to Bayes' rule, suppose we want to infer a parameter $\theta$ given training data $X$ and training label $Y$ in supervised learning: 
\begin{equation}
    \label{eqn:chap2-bayes-theorem}
    P(\theta | X, Y) = \frac{P(Y | X, \theta) P(\theta)}{P(Y | X)} = \frac{P(Y | X, \theta) P(\theta)}{\int P(Y | X, \theta) P(\theta ) \dby \theta}
\end{equation}
For the prediction task, given the input data point $x^*$ that we would like to make prediction $y^*$ of, we can do marginalization of all parameters i.e
\begin{equation}
    \label{eqn:chap2-bayes-pred}
    \int P(y^* | x^*, X, Y, \theta) \dby\theta = \int P(y^* | x^*, \theta) P(\theta | X, Y) \dby\theta
\end{equation}
The biggest problem in Bayesian inference is when the integral is intractable especially in the denominator of second equality in equation \ref{eqn:chap2-bayes-theorem}. Note that in equation \ref{eqn:chap2-bayes-pred}, we can approximate it via Monte-Carlo sample assuming we can sample $P(\theta | X, Y)$. In this thesis, we will focus on variational inference techniques, however, there are other techniques for example, conjugate prior.
% \Phu{Do i have to cite Learning in Graphical Models ?}

\subsection{Variational inference techniques}
\label{sec:chap2-vi-technique}
We would like to approximate the posterior $P(\theta | X, Y)$ with parametric distribution $q_{\phi} (\theta)$. This can be done by minimizing Kullback–Leibler divergence(KL-divergence)\footnote{Note that given 2 difference distributions $q$ and $p$, $\kl(q \| p) \ne \kl(p \| q)$.} between them:
\begin{equation}
    \kl\left( q_{\phi} (\theta) \Big\| P(\theta | X, Y) \right) = \int q_{\phi} (\theta) \log\left( \frac{q_{\phi} (\theta)}{P(\theta | X, Y)} \right) \dby \theta
\end{equation}
\textbf{Minimizing} the KL-divergence is equivalent to \textbf{maximized} the following value, which we will called it Evidence lower bound (ELBO) defined by
\begin{equation}
\begin{aligned}
    \argmax{\theta} \mathbb{E}_{q_{\phi} (\theta)} \left[ P(Y | X, \theta) \right] - \kl \left(q_{\phi} (\theta) \Big\| P(\theta) \right) &= \argmax{\theta} \int q_{\phi}(\theta)\log\left(\frac{P(Y, \theta | X)}{q_{\phi}(\theta)}\right) \dby \theta \\
    &= \argmin{\theta } \kl\left( q_{\phi} (\theta) \Big\| P(\theta | X, Y) \right)
\end{aligned}
\end{equation}
Given this can calculate this without relying on $P(Y | X)$, the intractable terms. The full derivation is in appendix \ref{appx:chap2-elbo-kl} for completeness. Furthermore, we can derive ELBO based on Jensen's inequality, which is shown in \ref{appx:chap2-elbo-jensen}.\footnote{We think that minimizing the KL-divergence makes more sense compared to Jensen's inequality. However, we still have to mention it because some of the key papers ultized this technique.}

% Now, we will step back and define the log-likelihood, note that we can also derive ELBO from this quantity using Jensen's inequality, see appendix \ref{appendix-1:elbo-jensen}
% \begin{equation}
%     l(X, Y) = \log P(Y| X) = \log \left( \int P(Y,  \theta | X) \dby \theta \right)
% \end{equation}
% % We will treat $\theta$ as hidden variable, and then finding how can we increase the likelihood of this, which will provides the reason behind the maximizing ELBO. 
% We can show the gap between this log-likelihood that we want to optimize and ELBO 
% \begin{equation}
%     l(X, Y) - \underbrace{\int q_{\phi}(\theta) \log\left(\frac{P(Y, \theta | X)}{q_{\phi}(\theta)}\right) \dby \theta}_{\text{ELBO}} = \int q_{\phi} (\theta) \log\left( \frac{q_{\phi} (\theta)}{P(\theta | X, Y)} \right) \dby \theta
% \end{equation}
% The derivation will be in appendix \ref{appendix-1:elbo-gap}. Interestingly, it is the same as the objective that we want to optimize earlier, which is KL-divergence between variational distribution and true posterior. We would like to note that minimizing this is possible, since it is shown earlier to be the same as maximizing ELBO but it is intractable to calculate its value. Finally, KL-divergence is always non-negative (this is known as Gibbs' inequality), therefore ELBO will stay lower bound, and optimizing ELBO will also optimize whole $l(X, Y)$

\subsection{Expectation Maximization (EM) Algorithm}
\label{sec:chap2-em-algo}
Now after we have explored the use of variational inference technique, we will now turn to EM algorithm. We will follows the example provided in \cite{fellows2019virel}. Suppose, we have a $N$ data points $X = \{x^{(n)}\}^N_{n=1}$ that we know that it is generated by each hidden variable $z = \brackc{z^{(n)}}^N_{n=1}$. We would like to learn the "process" that generates such a data point, i.e the variable $\theta$, such that $P_{\theta}(X | z)$. It is clear that we would like to find $\theta$ that maximizes the log-likelihood of the observed data point
\begin{equation}
    l_{\text{unsup}}(\theta) = \log \left( P_{\theta}(X) \right) = \log\bracka {\int P_{\theta}(X, z) \dby z}
\end{equation}
With this, we can also introduce the variational distribution over the posterior hidden variable (since we don't know what $P(z | X)$ is) $q_{\phi}(z)$, with similar pattern, we can decompose the marginal log-likelihood into:
\begin{equation}
    \begin{aligned}
        l_{\text{unsup}}(\theta) &= \int q_{\phi}(z) \log \bracka{\frac{P(X, z)}{P(z | X)}} \dby z \\
        &= \underbrace{\int q_{\phi}(z) \log \bracka{\frac{P_{\theta}(X, z)}{q_{\phi}(z)}} \dby z}_{\circled{1}} + \underbrace{\int q_{\phi}(z) \log \bracka{\frac{q_{\phi}(z)}{P_{\theta}(z | X)}} \dby z}_{\circled{2}}
    \end{aligned}
\end{equation}
We can show the equality above is true in \ref{appx:chap2-elbo-gap}. Now, if we look carefully we can see that part $\circled{2}$ is KL-divergence between $q_{\phi}(z)$ and $P_{\theta}(z | X)$. In theory, we can have the exact version of EM algorithm as follows: at step $k$, we first make the gap between part $\circled{1}$ and log-likelihood tight by setting the exact posterior to the variational distribution $q_{\phi}(z)$
\begin{equation}
    q_{\phi^{(k+1)}}(z) \leftarrow P_{\theta^{(k)}}(z | X)
\end{equation}
We call this \textit{Expectation-Step} or \textit{E-step}, then we can optimizing $\theta$ by maximizing the part $\circled{2}$ of the log-likelhood:
\begin{equation}
    \theta^{(k+1)} \leftarrow \argmax{\theta^{(k)}}  \int P_{\theta^{(k)}}(z | X) \log \bracka{\frac{P_{\theta^{(k)}}(X, z)}{P_{\theta^{(k)}}(z | X)}} \dby z 
\end{equation}
We can this \textit{Maximization-Step} or \textit{M-Step}. However, it is clear that in M-step we are maximizing ELBO with respect to $\theta$, while in E-step we maximizing ELBO with respect to $\phi$\footnote{Recall from section \ref{sec:chap2-vi-technique} that minimizing KL is same as maximizing ELBO}. Now, with this \correctquote{coincident}, we can perform variational EM\footnote{We are using variational distribution to approximate the posterior.} based on the following ELBO:
\begin{equation}
    \begin{aligned}
        \mathcal{L}(\theta, \phi) &= \int q_{\phi}(z) \log \bracka{\frac{P_{\theta}(X, z)}{q_{\phi}(z)}} \dby z
        = \mathbb{E}_{q_{\phi}(z)} \brackb{\log P_{\theta}(X | z)} - \kl\bracka{q_{\phi}(z) \Big\| P(z)}
    \end{aligned}
\end{equation}
Now, the training scheme for variational EM is as follows (at step $k$):
\begin{equation}
    \begin{aligned}
        &\text{E-Step:} \quad \phi^{(k+1)} \leftarrow \argmax{\phi^{(k)}}\mathcal{L}(\theta^{(k)}, \phi^{(k)}) \qquad \text{M-Step:} \quad \theta^{(k+1)} \leftarrow \argmax{\theta^{(k)}}\mathcal{L}(\theta^{(k)}, \phi^{(k+1)})
    \end{aligned}
\end{equation}

\subsection{Stein Variational Gradient Descent (SVGD) \cite{liu2016stein}}
SVDG is used in many of the algorithm that we are presenting (for example soft Q-learning \cite{haarnoja2017reinforcement} and PR2 \cite{wen2019probabilistic}). It is a \correctquote{general purpose variational inference algorithm} \cite{liu2016stein}, which can also be used to "sample" the data point from complex distribution (optimized variational distribution). This is extremely useful since we want to construct and use neural network based on variational distribution for our agents. 

%\grammar{The algorithm itself is simple, yet, having very interesting mathematical formulation behind it. We suggest interested reader to check out the paper. We will described the general form of the algorithm, while the details will be left for each of its usage.} 

Suppose that we want to sample the distribution\footnote{which is usually unnormalized i.e the denominator, similar to \ref{eqn:chap2-bayes-theorem} is intractable} $P$ by a set of $n$ particles $\{\xi^{(0)}_i\}^n_{i=1}$. Then we can update the set of such particles such that the final set of $n$ particles $\{\xi^{(T)}_i\}^n_{i=1}$ represent sampled point from the distribution $P$. The algorithm does the following update rule (for iteration at step $t$): 
\begin{equation}
    x^{(t+1)}_i \leftarrow x^{(t)}_i + \alpha \bracka{\frac{1}{n}\sum^n_{j=1} \brackb{k(x^{(t)}_j, x^{(t)}_i) \nabla_{x^{(t)_j}} \log P(x^{(t)}_j) + \nabla_{x^{(t)_j}} k (x^{(t)}_j, x^{(t)}_i) }}
\end{equation}
where $\alpha$ is the updating step and we can use other optimization methods, such as ADAM \cite{kingma2014adam}, to optimize the points. 

% \factcheck{The author has shown that this update rule follows the functional gradient of KL-divergence between the variational distribution ($q_{[T]}$ which is the density of $z = T(\xi)$) and the true distribution $P$ (Theorem 3.3). See the original paper for more details.}

% \grammar{This is used in variational auto-encoder \cite{kingma2013auto} where by we parameterized encoder as $q_{\phi}(z | X) = \mathcal{N}(z ; \mu_{\phi}(X), \Sigma_{\phi}(X))$, where $\mu_{\phi}$ and $\Sigma_{\phi}$ are neural network. For, the decoder as $P_{\theta}(X | z) = \mathcal{N}(X ; \mu_{\theta}(z), \Sigma_{\theta}(z))$. However, for decoder we usually ignore the covariance matrix, and for encoder, we tends to assume independences between each latent variables (diagonal covariance matrix) for tractable computation. Since we have both parameters available, we can train both neural networks at the same time by optimizing the ELBO.}
% \begin{equation}
%     \mathbb{E}_{q_{\phi}(z | X)} \brackb{\log P_{\theta}(X | z)} - \kl\bracka{q_{\phi}(z | X) \Big\| P(z)}
% \end{equation}
% \grammar{The authors used re-parametrisation trick (pathwise derivative \cite{gal2016uncertainty}) in order to optimize the first quantity in ELBO. Generally, suppose we want to find the gradient of $\mathbb{E}_{a \sim p_{\theta}}[f(a)]$. This can be done by finding differentiable function $g(\theta, \varepsilon)$, where $\varepsilon \sim \mathcal{X}$ and $\mathcal{X}$ is arbitrary distribution that we can sample, such that $\mathbb{E}_{\varepsilon \sim \mathcal{X}}\brackb{f(g(\theta, \varepsilon))} = \mathbb{E}_{a \sim p_{\theta}}[f(a)]$. For example, for normal distribution $\mathcal{N}(x; \mu, \sigma)$, we have $g(\mu, \sigma, \varepsilon) = \mu +\sigma\varepsilon$ where $\varepsilon \sim \mathcal{N}(\varepsilon;0, 1)$. The gradient, thus, can easily be computed by chain rule.}

\section{Single Agent Reinforcement Learning}
\label{sec:chap2-single-rl}
In this section, we will go through the classical notion of single agent reinforcement learning as a basis to the next section, which is about deep reinforcement learning - a scale and approximated version of algorithms discuss here.

\Phu{citing the courses and book ?}
\subsection{Markov Decision Process}
\label{sec:chap2-rl-defintions}
Markov Decision process (MDP) formulates the task that reinforcement learning trying to solve\footnote{It is a mathematical description of \correctquote{game}.}. 
\begin{definition}
    Markov decision process (MDP) is a tuple : $\langle S, A, R, T, p_0, \gamma \rangle$ where 
    \begin{itemize}
        \item State space: $S = \{s_0, s_1, \cdots , s_{|S|}\}$. Usually we have $s_i \in \mathbb{R}^{d_s}$
        \item Action space: $A = \{a_0, a_1, \cdots a_{|A|}\}$
        \item Reward function: $\mathcal{R}: S \times A \rightarrow \mathbb{R}$
        \item Transition function: $T: S \times A \times S \rightarrow [0, 1]$, where $s^{(t+1)} \sim T(s^{(t+1)} | s^{(t)}, a^{(t)})$ 
        \item Initial state distribution: $p_0 : S \rightarrow [0, 1] $, where $s^{(0)} \sim p_0$
        \item Discounted factor: $\gamma \in (0, 1]$
    \end{itemize}
    In almost all the cases, we include player's policy $\pi: S \rightarrow \Delta(A)$ where $\Delta(A)$ is probability simplex over $A$, where $a^{(t)} \sim \pi(a^{(t)} | s^{(t)})$ denote player choose action $a_t$ at state $s_t$ at time $t$. The goal of any reinforcement learning algorithms is to maximizes the reward with or without full description/experience of MDP.
\end{definition}
\noindent
We denote the value function of the player's policy $V^\pi(s)$ as the expected future cumulative reward given the state the agent is in:
\begin{equation}
    V^\pi(s) = \mathbb{E}_{a_t \sim \pi, s_t \sim T}\brackb{\sum^\infty_{t=0} \gamma^t r(s_t, a_t) \Bigg| s_0 = s}
\end{equation}
And the action value function is defined as which is the expected future reward when agent perform action $a$ at state $s$: 
\begin{equation}
    Q^\pi(s, a) = \mathbb{E}_{a_t \sim \pi, s_t \sim T}\brackb{\sum^\infty_{t=0} \gamma^t r(s_t, a_t) \Bigg| s_0 = s, a_0 = a}
\end{equation}
The advantage function of an agent is simply the differences between value function and action value function i.e $A^{\pi}(s, a) = Q^\pi(s, a) - V^\pi(s)$. Furthermore, We can establish the connection between both $V(s)$ and $Q(s, a)$ functions as
\begin{equation}
    \label{eqn:chap2-Q-and-V}
    Q^\pi(s, a) = r(s, a) + \gamma \mathbb{E}_{s' \sim T(s' | s, a)}\brackb{V^\pi(s')} \quad \quad \quad V^\pi(s) = \mathbb{E}_{a \sim \pi(a | s)} \brackb{Q^\pi(s, a)}
\end{equation}
We denote the optimal value function and optimal action value function as $V^*(s)$ and $Q^*(s, a)$ when $V^{\pi^*}(s)$ and $Q^{\pi^*}(s, a)$, respectively, where $\pi^* \in \arg\max_{\pi} V^{\pi}(s)$ for any state $s$. Now, we want to establish connection between these 2 optimal functions, which is:
\begin{equation}
\label{eqn:chap2-optim-Q-and-V}
    Q^*(s, a) = r(s, a) + \gamma \mathbb{E}_{s' \sim T(s' | s, a)}\brackb{V^*(s')} \quad \quad \quad V^*(s) =\max_{a\in A} Q^*(s, a)
\end{equation}
We can now slightly change the objective of reinforcement learning to \correctquote{How to find such an optimal policy}.

\subsection{Policy Iteration}
\label{sec:chap2-policy-iter}
We will now introduce the first algorithm that will solve the problem of reinforcement learning, called policy iteration. The strategy is based on simple observation: the agent will always doing better or equal if we set it's behavior to be following the optimal value function in equation \ref{eqn:chap2-optim-Q-and-V} i.e we can improve our policy $\pi$ by greedily chooses the action that would maximized expected future rewards given $Q^\pi$:
\begin{equation}
    \label{eqn:chap2-greedy-Q}
    \pi'(a | s) = \begin{cases}
        1 &\text{ if } a = \argmax{a \in A} Q^\pi(s, a) \\
        0 &\text{ otherwise }
    \end{cases}
\end{equation}
we can call this step, a policy improvement step. One can show that $\forall s \in S: V^\pi(s) \le V^{\pi'}(s)$. Furthermore, we can show that we will reach optimal value function if we keep improving the policy. The proof, for completeness, will be presented in appendix \ref{appx:chap2-rl-policy-improve}. What we have left is to find an algorithm that gives us value function $V^{\pi}(s)$ for any policy $\pi$, which we will call this step \textit{policy evaluation}. We can expanded the equality in equation \ref{eqn:chap2-Q-and-V} as:
\begin{equation}
    V^\pi(s) = \mathbb{E}_{a \sim \pi(a | s)} \brackb{r(s, a) + \gamma \mathbb{E}_{s' \sim T(s' | s, a)}\brackb{V^\pi(s')}} 
\end{equation}
We call this equation expected Bellman equation along with the following expected Bellman operator $\contractop^{\pi} : \mathbb{R}^{|S|} \rightarrow \mathbb{R}^{|S|}$ defined as:
\begin{equation}
    \label{eqn:chap2-exp-bellman-operator}
    \contractop^{\pi} V(s) = \mathbb{E}_{a \sim \pi(a | s)} \brackb{r(s, a) + \gamma \mathbb{E}_{s' \sim \contractop(s' | s, a)}\brackb{V(s')}} 
\end{equation}
To solve the policy evaluation problem, we want to find $V(s)$ such that $\contractop^\pi V(s) = V(s)$ however, to solve this via matrix form would be too expensive. However, turn out this expected Bellman operator is an contraction mapping on $\infty$-norm i.e 
\begin{equation*}
    \| \contractop^\pi V_1(s) - \contractop^\pi V_2(s) \|_\infty \le \alpha \|V_1(s) - V_2(s)\|_\infty
\end{equation*}
where $\|V(s)\|_\infty = \max_{s \in S} V(s)$. For some $0 \le \alpha < 1$ (See appendix \ref{appx:chap2-rl-expected-bell-contract} for proof), then by Banach fixed-point theorem
\begin{theorem}{(Banach fixed-point theorem \cite{murfet_2019})}
    Given the complete (every Cauchy sequence converges to a point in that space) metric space $(X, d(\cdot, \cdot))$ and the contraction mapping $\contractop : X \rightarrow X$, the sequence $(\contractop^{(n)} x)^{\infty}_{n=1}$ converges to a fixed point $x^*$ where $\contractop x^* = x^*$, for all points $x \in X$. 
\end{theorem}
repeatedly apply the Bellman operator will lead us to a solution of $V^{\pi}$.  In conclusion, Policy iteration is an algorithm that solves the MDP by involving the alternation between the following 2 steps
\begin{enumerate}
    \item \textbf{Policy Evaluation}: Starting with randomized value function and Repeatedly applying Bellman operator defined in equation \ref{eqn:chap2-exp-bellman-operator} until converge to get true value function $V^{\pi}(s)$.
    \item \textbf{Policy Improvement}: Update the policy by choosing the best action from action value function (we can calculate this by the equation \ref{eqn:chap2-Q-and-V}) following the equation \ref{eqn:chap2-greedy-Q}.
\end{enumerate}
Since the value function always increases in every state, this algorithm is guaranteed to reach the optimal value function and policy. 

\subsection{Value Iteration}
\label{sec:chap2-value-iter}
Now, it is quite computational ineffective to evaluates the policy every times to update the policy. Can we just update the value function using single Bellman update and then uses the policy improvement to come-up with the next value function? The answer is \textit{Yes}. We define optimal value Bellman update operator as:
\begin{equation}
    \label{eqn:chap2-optimal-bellman-operator}
    \contractop^*_V V(s) = \max_{a \in A}\Big[r(s, a) + \gamma \mathbb{E}_{s' \sim \mathcal{T}(s' | s, a)}\brackb{V(s')} \Big]
\end{equation}
Note that it is related to optimal Bellman equation represented in equation \ref{eqn:chap2-optim-Q-and-V}. In this case, we update the the value function toward the action that yields highest value given only one update. This is, indeed, a contraction mapping, thus repeatedly updating the value function by optimal value Bellman operator will give us optimal value function. The prove will be appendix \ref{appx:chap2-rl-optimal-val-bellman-contract}. Finally, given the optimal value function, one can calculate the optimal action value function, and optimal policy. 

\subsection{Q Iteration}
\label{sec:chap2-Q-iter}
Now, instead of using value function as in value iteration, it is always desirable for us to estimate optimal action value function $Q^*(s, a)$ direction and without having to calculate transition function expectation. By using the same technique, one can define the optimal action value Bellman operator as 
\begin{equation}
    \contractop^*_Q Q(s, a) = r(s, a) + \gamma \mathbb{E}_{s' \sim \mathcal{T}(s' | s, a)}\brackb{\max_{a\in A} Q(s', a)} 
\end{equation}
We can proof in appendix \ref{appx:chap2-rl-optimal-q-bellman-contract} that this operator is indeed a contraction mapping. However, there is more to this. Since there is no need to explicitly consider the transition function, we defined the following update rule given iteartion $k$ and $(s_t, a_t)$ experience:
\begin{equation}
\label{eqn:chap2-Q-stochastic-update}
    Q^{(k+1)}(s_t, a_t) \leftarrow (1-\alpha_t) Q^{(k)}(s_t, a_t) + \alpha\brackb{r(s_t, a_t) + \gamma \max_{a'} Q^{(k)}(s_{t+1}, a')}
\end{equation}
where $0 < \alpha < 1$. Following the theorem 1 of \cite{jaakkola1994convergence}, which is also a basis of proofing Q-Iteration in later section states that:
\begin{theorem}
\label{thm:updaate-stochastic}{\cite{jaakkola1994convergence}}
    The process $\Delta_{t+1} = (1-\alpha_t(x)) \Delta_t + \alpha_t(x) F_n(x) $ will converge to zero if and only if The state space is finite, $\sum_t \alpha_t(x) = \infty$, $\sum_t \alpha_t(x)^2 < \infty$, $\left\|\mathbb{E}\brackb{F_n(x) \big| P_n} \right\| \le \beta \|\Delta_n\|$ where $\beta \in (0, 1)$ and  $\operatorname{Var}\brackb{F_n(x) \big| P_n} \le C\bracka{1 + \|\Delta_n\|}^2$, where $P_n$ is a collection of histories. 
\end{theorem}
By subtracting optimal Q function at both ends i.e $\Delta_t = Q^{(k)}(s_t, a_t) - Q^*(s_t, a_t)$ and $F_n(x) = r(s_t, a_t) + \gamma \max_{a'} Q^{(k)}(s_{t+1}, a') - Q^*(s_t, a_t)$, we can proof that the update in equation \label{eqn:chap2-Q-stochastic-update} will converge to the optimal Q-value. Given Theorem \ref{thm:updaate-stochastic}, we can now change any contraction mapping into a stochastic update rule and still maintains the convergence property. We will use this throughout the text.

\section{Deep Reinforcement Learning}
\label{sec:chap2-deep-rl}
Now, it is always the case that the total number of states is too big for us to represent the exact value function or action-value function. We will now present the way that we can approximate these values, together with another method for control, notably policy gradient and the combination of both paradigms: Actor-Critic algorithm.

\subsection{Deep Q-learning \cite{mnih2015human}}
Deep Q-learning is based on traditional Q-iteration, whereby instead of using stochastic iteration (see section \ref{sec:chap2-Q-iter}), the authors try to approximate optimal Q-function using neural network $Q_\theta(s, a)$ by training it to minimize mean square error between its prediction and the target prediction i.e
\begin{equation}
    \mathcal{L}(\theta) = \mathbb{E}_{(s, a, r, s')^n_{i=1} \sim B(\mathcal{D})}\brackb{ \frac{1}{n}\sum^n_{i=1} \bracka{Q_{\theta}(s_i, a_i) - r(s_i, a_i) - \gamma\max_{a'} Q_{\theta^-}(s'_i, a'_i)}^2 }
\end{equation}
The training relies on experience replay $\mathcal{D}$, which stores the past experiences in the tuple, and target action-value function $Q_{\theta'}(s, a)$ added to increase stability in training. The following process updates the target action-value function at time step $t$
\begin{equation}
\label{eqn:chap2-update-target}
    \theta^-_{t+1} \leftarrow \rho \theta^-_t + (1-\rho) \theta_t
\end{equation}
which is a common way to update a target function. $\mathcal{B}$ is a mini-batch sampler. Furthermore, due to the size of state space, the authors use $\varepsilon$-greedy policy to help the agent collecting the experience (better exploration) so that it gets more diverse set of training data. The policy is defined as
\begin{equation}
    \pi(a | s) = \begin{cases}
        \argmax{a'} Q(s, a') &\text{ with probability } \varepsilon \\
        a' \sim \text{Uniform}(a_0, a_1, \dots, a_{|A|}) &\text{ otherwise } \\
    \end{cases} \\
\end{equation}
and $\varepsilon$ decays overtime. The authors showed that given the following training scheme (with some minor tricks), the agent could achieve super-human performance on Arcade Learning Environment \cite{bellemare2013arcade}. Note that there are some games that the agent doesn't learn at all. The infamous Montezuma's revenge is one instance of this. Because the agent has to learn to reason over long time steps to solve the game, Q-learning isn't sufficient enough for reasoning over the long time step. This problem can be mitigated via hierarchical reinforcement learning \cite{kulkarni2016hierarchical, dayan1993feudal, vezhnevets2017feudal} or intrinsic motivation \cite{pathak2017curiosity}. 

% \footnote{This is one of the main reasons for the development in \ref{chapter:chap4}}.

Other techniques that improve the training stability or increase its performance are, notable, Double-Q learning \cite{van2016deep}, which tackles the over-optimism prediction by using the action output of \textit{current iteration} of agent estimate the value of target network in the target i.e
\begin{equation}
    \mathcal{L}(\theta) = \mathbb{E}_{(s, a, r, s')^n_{i=1} \sim B(\mathcal{D})}\brackb{ \frac{1}{n}\sum^n_{i=1} \bracka{Q_{\theta}(s_i, a_i) - r(s_i, a_i) - Q_{\theta^-}(s'_i, \argmax{a'} Q_\theta(s'_i, a'))}^2 }
\end{equation}
Furthermore, Prioritized DQN \cite{schaul2015prioritized}, which allows the experience replay to sample higher quality experience tuple, Dueling DQN \cite{wang2015dueling}, which increase the capacity of the estimation of action-value function by combining approximate value function and approximate advantage function, Distributional DQN \cite{bellemare2017distributional}, which estimates the distribution over the action-value function \cite{osband2018randomized}, Noisy Net \cite{fortunato2017noisy}, which injects the noise into the output of neural network layers, aiming to improves explorations, and combination of them all - rainbow DQN \cite{hessel2018rainbow}.

\subsection{Policy Gradient \cite{sutton2000policy}}
Now, instead of approximating value function, why don't we directly optimized $\pi_\theta(a | s)$ given the sum of expected reward as objective i.e 
\begin{equation}
    \pi_\theta^*(a | s) = \argmax{\theta} \eta(\pi_\theta) = \argmax{\theta}\mathbb{E}_{a_t, s_t \sim \pi_\theta}\brackb{\sum^\infty_{t=0}\gamma^t r(s_t, a_t)}
\end{equation}
The strategy is to find the gradient of the sum of expected reward, which we can state that it is equal to
\begin{equation}
    \frac{\partial}{\partial \theta} \mathbb{E}_{a_t, s_t \sim \pi_\theta}\brackb{\sum^\infty_{t=0}\gamma^t r(s_t, a_t)} = \mathbb{E}_{a_t, s_t \sim \pi_\theta}\brackb{R_t\frac{\partial}{\partial \theta} \log \pi_\theta(a_t | s_t) }
\end{equation}
where $Q^\pi(s_t, a_t) = \mathbb{E}\left[R_t | s_0 = s_t, a_0 = a_t\right]$. Now, we can perform stochastic approximation via Monte Carlo sampling, while we can train a neural network using stochastic gradient ascent. We call this basic algorithm REINFORCE \cite{sutton2000policy}. However, this algorithm is notoriously hard to train due to the high variance of the estimator. We can reduce the variance by minus the Q-value estimation by a baseline $B(s)$ value that is independent of $\theta$ and $a$. In usual case, we use value function estimator $B_t$\footnote{where $V^\pi(s_t) = \mathbb{E}\brackb{B_t | s_0 = s_t}$}, which leads us to use of advantage function $A^\pi(s, a)$ estimator\footnote{This would make sense intuitively because advantage function only estimates how good each action is given the current state, which is the only measurement we want for policy gradient algorithm.}. Furthermore, we can add an entropy regularizer to help the policy learns via maximum entropy principles, see section \ref{sec:chap1-MERL-intro} for a quick introduction.

\subsection{Actor-Critic}
\label{sec:chap2-ator-critic}
Now, instead of using a Monte Carlo estimation of the reward, why don't we try to estimate this quality using function approximator like a neural network? By using a neural network approximation, we now move toward Actor-Critic algorithm; hence, the name \correctquote{Advantage Actor-Critic Algorithms}. \cite{mnih2016asynchronous}\footnote{Also see  \url{https://openai.com/blog/baselines-acktr-a2c/}}. The usual implementation is as follows: for the actor/policy network, the implementation is similar. But for critic/advantage function estimator, we only need to train value function $V_\theta(s_t)$  and estimate the advantage function:
\begin{equation}
    A(s_t, a_t) \approx \underbrace{\sum^n_{i=0} \gamma^i r_{t + i} + \gamma^{n+1} V_\theta(s_{n+1})}_{\approx Q_\theta(s_t, a_t)} - V_\theta(s_t)
\end{equation}
The longer the $n$ times steps, the more accurate (and more variance) the estimator going to become. We can use this value to train the agent. Now, we can train the value function by minimizing the following objective:
\begin{equation}
\label{eqn:chap2-ac-critic-loss}
    \loss(\theta) = \mathbb{E}_{(r_0, \cdots, r_{n+1}), s_{n+1} \sim \mathcal{D}}\brackb{\frac{1}{2}\bracka{V_\theta(s_0) - \sum^n_{i=0} \gamma^i r_i - \gamma^{n+1}V_\theta(s_{n+1})}^2}
\end{equation}
which is a mean-square error between predictive value and bootstrapped\footnote{Bootstrap is the act that we include $V_\theta(s_{n+1})$ after a certain length of cumulative reward instead of having full trajectory, i.e. using one prediction as part of the target for other prediction} value. Now, we have laid the essential foundation to Actor-Critic (AC) method let's consider two variances of successful extension to AC, which are TRPO \cite{schulman2015trust} and PPO \cite{schulman2017proximal}. Let's start with TRPO, which is based on the following theorem presented in \cite{schulman2015trust}:
\begin{theorem}
Let $\kl^{\max}\bracka{\pi, \tilde{\pi}} = \max_s \kl\bracka{\pi(\cdot | s) \big\| \tilde{\pi}(\cdot | s)}$  then the following holds
\begin{equation}
    \eta(\tilde{\pi}) \ge L_\pi(\tilde{\pi}) - \frac{4\varepsilon \gamma}{(1-\gamma)^2}\kl^{\max}\bracka{\pi, \tilde{\pi}}
\end{equation}
where $\varepsilon = \max_{s, a}|A_\pi(s, a)|$ and $L_\pi(\tilde{\pi})$ is defined as:
\begin{equation}
    L_\pi(\tilde{\pi}) = \eta(\pi) + \sum_s \rho_\pi(s) \sum_a \tilde{\pi}(a | s) A_\pi(s, a)
\end{equation}
As the visitation frequency $\rho_\pi(s) = \sum^\infty_{n=0} \gamma^n P(s_n = s)$
\end{theorem}
Given $M_i(\pi) = L_{\pi_i}(\pi) - C\kl^{\max}\bracka{\pi_i, \pi}$, we have the inequality $\eta(\pi_{i+1}) \ge M_i(\pi_{i+1})$ and $\eta(\pi_i) = M_i(\pi_i)$, this leads to $\eta(\pi_{i+1}) - \eta(\pi_i) \ge  M_i(\pi_{i+1}) - M_i(\pi_i)$. This implies that if we optimize $M_i(\pi)$, then we would optimize $\eta(\pi)$. Now, we can turn $M_i(\pi)$ into constrained optimization problem 
\begin{equation}
\begin{aligned}
    &\max_\theta L_{\theta_\text{old}}(\theta) \\
    \text{ such that } &\kl^{\max}\bracka{\theta_{\text{old}}, \theta} \le \delta
\end{aligned}
\end{equation}
The problem is that the KL-divergence might not be tractable. We can approximate this as $D^\rho_{\text{KL}}(\theta_1, \theta_2) = \mathbb{E}_{s\sim\rho}\brackb{\kl\bracka{\pi_{\theta_1}(\cdot | s) \big\| \pi_{\theta_2}(\cdot | s)}}$. Furthermore, the authors also proposed the importance sampling scheme, the objective becomes 
\begin{equation}
\begin{aligned}
    &\mathbb{E}_{s \sim \rho_{\theta_{\text{old}}}, a \sim \pi_{\theta_{\text{old}}}}\brackb{\frac{\pi_\theta(a | s)}{\pi_{\theta_{\text{old}}}(a | s)} A_{\pi_{\theta_{\text{old}}}}(s, a)} \\
    \text{ such that }& D^{\rho_{\theta_\text{old}}}_{\text{KL}}(\theta_{\theta_\text{old}}, \theta) \le \delta
\end{aligned}
\end{equation}
The trust region is the KL-constraint imposed on policy optimization by not allowing the updated policy to be too far from the current policy. All the proofs and implementations are explored in details in \cite{schulman2015trust}. PPO \cite{schulman2017proximal} is the approximate version of TRPO. There are two variances for PPO that are proposed in the paper:
\begin{itemize}
    \item Clipped Surrogate Objective: Using the following object for the objective
    \begin{equation}
        \mathbb{E}\brackb{\min\bracka{\frac{\pi_\theta(a | s)}{\pi_{\theta_{\text{old}}}(a | s)} A_t, \operatorname{clip}\bracka{\frac{\pi_\theta(a | s)}{\pi_{\theta_{\text{old}}}(a | s)}, 1-\varepsilon, 1+\varepsilon} A_t}}
    \end{equation}
    where $\varepsilon$ is hyper-parameter and usually set to $0.2$
    \item  Adaptive KL Penalty Coefficient: We consider the unconstrained version of KL-divergence and update its weight on KL-divergence $\beta$ based on the current evaluation:
    \begin{equation}
        \begin{aligned}
            &\mathbb{E}\brackb{\frac{\pi_\theta(a | s)}{\pi_{\theta_{\text{old}}}(a | s)} A_t - \beta \kl\bracka{ \pi_{\theta_\text{old}}(\cdot | s) \big\| \pi(\cdot | s)}} \text{ where } \beta \leftarrow \begin{cases}
                \beta/2 &\text{ if } d < d_{\operatorname{target}}/1.5 \\
                \beta * 2 &\text{ if } d > d_{\operatorname{target}} * 1.5 \\
                \beta &\text{ otherwise }
            \end{cases}
        \end{aligned}
    \end{equation}
\end{itemize}

\subsection{Deep Deterministic Policy Gradient}
\label{sec:chap2-ddpg}
Now, we will consider the problem in which the agent's action is continuous. The resulting algorithm based on Actor-Critic variance called Deep Deterministic Policy Gradient (DDPG) \cite{lillicrap2015continuous}. We will also consider the extension to Twin Delayed DDPG (TD3) \cite{fujimoto2018addressing} to make the training of DDPG more stable. Starting with DDPG, the core intuition behind it\footnote{There are more into the theory of how it works. We suggest reading \cite{silver2014deterministic}} is that we train the agent to maximize critic's prediction of reward, i.e. maximizing the following objective with respect to $\theta$
\begin{equation}
    \loss(\theta) = Q_\phi(s, \pi_\theta(s))
\end{equation}
Note that we didn't do any sampling the agent, hence the name \textit{deterministic}.  The actor is training with typical mean-square error loss, similar to equation \ref{eqn:chap2-ac-critic-loss}:
\begin{equation}
    \loss(\phi) = \mathbb{E}_{\mathcal{D}}\brackb{\frac{1}{2}\bracka{Q_\phi(s, a) - r - \gamma Q_{\phi^-}(s', \pi_\theta(s'))}^2}
\end{equation}
Now, TD3 extends DDPG algorithms as follows
\begin{itemize}
    \item Using double-Q learning. By training the 2 critics together with the following target (hence the name \correctquote{twin})
    \begin{equation}
    \begin{aligned}
        &r + \gamma \min_{i \in \brackc{1, 2}} Q_{i, \phi^-}(s', a') \\
        \text{ where }& a' = \text{clip}\bracka{\pi_{\theta^-}(s'), \text{clip}(\varepsilon, -c, c), a_\text{low}, a_\text{high}} \qquad \varepsilon \sim \mathcal{N}(0, \sigma)
    \end{aligned}
    \end{equation}
    \item We update actor network in lesser frequency than the critic (this includes the target actor, which is updated similar to equation \ref{eqn:chap2-update-target})
\end{itemize}
This two techniques will be a trick for other algorithms to improve the performance and stability.

\section{Control as Inference Framework}
\subsection{Derivations for closed form solution}
\label{sec:chap2-derivation-soft-closed-form}

We will base the derivation on \cite{levine2018reinforcement}. The derivation of this will be the basis for \textit{all} derivations of the algorithms in this thesis; therefore, we will state full proof within the section. As introduced in section \ref{sec:chap1-MERL-intro}, we would like to find the posterior where the agent is optimal. Let's recall the definition of optimality. 
\begin{equation}
    P(\mathcal{O}_t = 1 | s_t, a_t) \propto \exp (\beta r (s_t, a_t))
\end{equation}
Let's consider the graphical model representation for joint distribution, which is in figure \ref{fig:chap2-single-graphical-optim}. 
\begin{figure}[ht]
    \begin{minipage}[t]{0.5\linewidth}
    \centering
    \begin{tikzpicture}[latent/.append style={minimum size=1.0cm}]
        \node[obs] (ot) {$\mathcal{O}_t$};
        \node[latent, below=of ot] (at) {$a_t$};
        \node[latent, below=of at] (st) {$s_t$};
        
        % \path[->]  (o1)  edge   [bend left=45] node {} (a);
        \edge {st} {at}
        \edge {at} {ot}
        \path[->]  (st)  edge   [bend left=30] node {} (ot);
        
        \node[obs, right=3cm of ot] (ot1) {$\mathcal{O}^i_{t+1}$};
        \node[latent, below=of ot1] (at1) {$a_{t+1}$};
        \node[latent, below=of at1] (st1) {$s_{t+1}$};

        \edge {st1} {at1}
        \edge {at1} {ot1}
        \path[->]  (st1)  edge   [bend right=30] node {} (ot1);
        
        \path[->]  (st)  edge   [] node {} (st1);
        \path[->]  (at)  edge   [] node {} (st1);
        
        \node[latent, right=3cm of st1] (stn) {$s_{t+n}$};
        \path[->]  (st1)  edge [] node {} (stn);
    \end{tikzpicture}
    \end{minipage}%
    \begin{minipage}[t]{0.5\linewidth}
    \caption{The graphical model representation of joint probability that we want to approximate for single agent reinforcement learning.}
    \label{fig:chap2-single-graphical-optim}
    \end{minipage}
\end{figure}
The joint probability is equal to
\begin{equation}
\label{eqn:chap2-joint-single-optim}
    P(s_{1:T}, a_{1:T}, \mathcal{O}_{1:T} = 1) = P(s_0) \prod^{T}_{t=1} P(s_{t+1} | s_t, a_t) P_{\prior}(a_t | s_t) P(\mathcal{O}_t = 1 | s_t, a_t)
\end{equation}
Now, we want to approximate this joint probability using the following graphical model, which is depicted in figure \ref{fig:chap2-single-graphical-approx}.
\begin{figure}[ht]
    \begin{minipage}[t]{0.5\linewidth}
    \centering
    \begin{tikzpicture}[latent/.append style={minimum size=1.0cm}]
        \node[latent] (at) {$a_t$};
        \node[latent, below=of at] (st) {$s_t$};
        
        \edge {st} {at}
        
        \node[latent, right=3cm of at] (at1) {$a_{t+1}$};
        \node[latent, below=of at1] (st1) {$s_{t+1}$};

        \edge {st1} {at1}
        
        \path[->]  (st)  edge   [] node {} (st1);
        \path[->]  (at)  edge   [] node {} (st1);
        
        \node[latent, right=3cm of st1] (stn) {$s_{t+n}$};
        \path[->]  (st1)  edge [] node {} (stn);
    \end{tikzpicture}
    \end{minipage}%
    \begin{minipage}[t]{0.5\linewidth}
    \caption{The graphical model representation of the variational joint probability that we will use to approximate optimal join probability in represented in equation \ref{eqn:chap2-joint-single-optim}. }
    \label{fig:chap2-single-graphical-approx}
    \end{minipage}
\end{figure}
Now, the variational joint probability is equal to\footnote{see the equation below}
\begin{equation}
    q(s_{1:T}, a_{1:T}) = P(s_0) \prod^{T}_{t=1} P(s_{t+1} | s_t, a_t) \pi(a_t | s_t) 
\end{equation}
Let's try to maximizing the negative KL-divergence between 2 probability distribution:
\begin{equation}
\begin{aligned}
    -\kl\bracka{q(s_{1:T}, a_{1:T}) \Big\| P(s_{1:T}, a_{1:T} | \mathcal{O}_{1:T} = 1)} &= -\mathbb{E}_{q(s_{1:T}, a_{1:T})}\brackb{\log\frac{q(s_{1:T}, a_{1:T})}{P(s_{1:T}, a_{1:T} | \mathcal{O}_{1:T} = 1)}} \\
    &=\mathbb{E}_{q(s_{1:T}, a_{1:T})}\brackb{\log\frac{P(s_{1:T}, a_{1:T} | \mathcal{O}_{1:T} = 1)}{q(s_{1:T}, a_{1:T})}} \\
    &= \mathbb{E}_{q(s_{1:T}, a_{1:T})} \brackb{\sum^{T}_{t=1} \beta r(s_t, a_t) - \log \frac{\pi(a_t | s_t)}{P_{\prior}(a_t | s_t)} }
\end{aligned}
\end{equation}
The full expansion will be in the appendix \ref{appx:chap2-ELBO-KL-Single} with alternative Jensen's inequality derivation in appendix \ref{appx:chap2-ELBO-Jensen-Single} for completeness\footnote{We have shown in section \ref{sec:chap2-vi-technique}, and this is going to be the last derivation with Jensen's inequality}, and please note that $\tau = (s_{1:T}, a_{1:T})$. We can use a policy gradient to help us on optimizing the objective: that is fine. But, let's consider the closed-form solution to this problem. We are trying to solve it from the last time step and then inductively solving the equation to general time step $t$. Start with the final time step $T$, which we want to maximize the following objective:
\begin{equation}
    \mathbb{E}_{q} \left[\beta r(s_T, a_T) - \log \frac{\pi(a_T | s_T)}{P_{\prior}(a_T | s_T)}\right]
\end{equation}
We will introduce $V(s_T)$; however, its value will be clear after the derivation: 
\begin{equation}
    \mathbb{E}_{q} \left[\beta r(s_T, a_T) - \log \frac{\pi(a_T | s_T)}{P_{\prior}(a_T | s_T)} + V(s_T) - V(s_T) \right]
\end{equation}
Now, expand and rearrange the equation, which leads to:
\begin{equation}
    \mathbb{E}_q \left[ V(s_T) \right] - \mathbb{E}_q \left[ \kl\left( \pi(a_T | s_T) \bigg|\bigg| \frac{\exp(\beta r(s_T, a_T)) P_{\prior}(a_T | s_T)}{\exp(V(s_T))} \right) \right]
\end{equation}
If we want to maximize the objective, which is the same as minimizing KL-divergence, we can set the policy to be 
\begin{equation}
    \label{eqn:chap2-soln-policy-terminal}
\begin{aligned}
    &\pi(a_T | s_T) = \frac{\exp(\beta r(s_T, a_T)) P_{\prior}(a_T | s_T)}{\exp(V(s_T))} \\
    \text{ where }& V(s_T) = \log \int \exp(\beta r(s_T, a_T)) P_{\prior}(a_T | s_T) \dby a_T
\end{aligned}
\end{equation}
We can see that $V(s_T)$ is a normalizing factor for a probability distribution. Note that it is only the case where the agent is in the form of Boltzmann policy that the value function is the normalizing factor. By setting the policy $\pi$ to minimize the KL-divergence, the objective becomes $V(s_T)$, which we would call it \correctquote{backward message} and we pass it down to the time step before, where we maximize the following objective with backward messages $\mathbb{E}_{s_{t+1}}[V\bracka{s_{t+1}}]$ with discounted factor as:
\begin{equation}
    \mathbb{E}_{q} \Bigg[\underbrace{\beta r(s_t, a_t) + \gamma\mathbb{E}_{s_{t+1}}[V\bracka{s_{t+1}}]}_{Q(s_t, a_t)} - \log \frac{\pi(a_t | s_t)}{P_{\prior}(a_t | s_t)} + V(s_t) - V(s_t) \Bigg] 
\end{equation}
Using the same method, we have the following general time policy $t$.We, sometimes, call this kind of policy a Boltzman Policy:
\begin{equation}
\begin{aligned}
\label{eqn:chap2-optimal-policy-general}
    &\pi(a_t | s_t) = \frac{\exp(Q(s_t, a_t)) P_{\prior}(a_t | s_t)}{\exp(V(s_t))} \\
    \text{ where }&V(s_t) = \log \int \exp(Q(s_t, a_t)) P_{\prior}(a_t | s_t) \dby a_t
\end{aligned}
\end{equation}
while we have Bellman like equation (see equation \ref{eqn:chap2-Q-and-V} for comparison)
\begin{equation}
    Q(s_t, a_t) = \beta r(s_t, a_t) + \gamma\mathbb{E}_{s_{t+1}}[V(s_{t+1})]
\end{equation}
Now, let's consider the meaning of for $V(s_{t})$, \cite{levine2018reinforcement} consider the \correctquote{value function}\footnote{In our case, it acts like a value function, however, for now (before showing a sufficient reason), we will consider it only to be a normalizing factor.} to be a \correctquote{soft-max}\footnote{This isn't the same \textit{softmax} as the one used in classification. However, it has the same property as maximize if $\beta \rightarrow \infty$, which is proven in \cite{fellows2019virel} } of the action value function. In \cite{leibfried2017information}, the authors changes the value of $\beta$ to balance between traditional deep Q-learning (when $\beta \rightarrow \infty$) and expected value based on prior i.e $\mathbb{E}_{a\sim P_\prior}\brackb{Q(s, a)}$ (when $\beta \rightarrow 0$) in order to mitigate the problem of over-estimation of deep Q-learning. Furthermore, in \cite{haarnoja2018softb}, the authors has consider automatic $\beta$ adjustment via constrained optimization. Let's further consider other form of $V(s_t)$, which we re-arrange the policy function:
\begin{equation}
\label{eqn:chap2-soft-value-def-expand}
    V(s_t) = \mathbb{E}_{a_t \sim \pi(a_t | s_t)}\brackb{Q(s_t, a_t) - \log \frac{\pi(a_t | s_t)}{P_{\prior}(a_t | s_t)}}
\end{equation}
Note that, we consider the policy to be the one represented in equation \ref{eqn:chap2-optimal-policy-general}\footnote{Some of the mistakes made comes from the consideration of expected $Q$ function without regularized terms for value function.}. Now, substitute the Q-function into the value function, which we have the recursive definition of value function:
\begin{equation}
    V(s_t) = \mathbb{E}_{a_t \sim \pi(a_t | s_t)} \left[\beta r(s_t, a_t) - \frac{\pi(a_t | s_t)}{P_{\prior}(a_t | s_t)} + \gamma\mathbb{E}_{s_{t+1}}[V(s_{t+1})]\right]
\end{equation}
Which we can roll out as the sum of regularized reward. This equality, therefore, has established the connection between this value function and reinforcement learning value function. Before we move to the implementation, let's consider some theoretical property of this Bellman equation. First, we can construct the soft Bellman operator as follows: 
\begin{equation}
    \contractop^{\pi} Q(s_t, a_t) = r(s_t, a_t) + \gamma \mathbb{E}_{s_{t+1} \sim p_{s}}\left[ \log \int \exp(Q(s_{t+1}, a)) P_{\prior}(a | s_{t+1}) \dby a \right]
\end{equation}
which we can prove that it is a contraction mapping, see appendix \ref{appx:chap2-soft-bellman-contract} for the proof, which is a restatement from \cite{haarnoja2017reinforcement}, while this will be useful later. Furthermore, we can show that the update of the policy leads to an increase in Q-function, which is proven in \cite{haarnoja2018softa} and will be restated and proof here. 
\begin{theorem}
    Let the updated policy $\pi_{\text{new}}$ be 
    \begin{equation}
        \pi_{\text{new}} = \argmin{\pi' \in \Pi} \kl\left( \pi'(\cdot | s_t) \bigg|\bigg| \frac{\exp(Q^{\pi_{\text{old}}}(s_t, \cdot)) P_{\prior}(\cdot | s_t)}{V^{\pi_{\text{old}}}(s_t)} \right)
    \end{equation}
    Then $\forall s_t, a_t : Q^{\pi_{\text{new}}}(s_t, a_t) \ge Q^{\pi_{\text{old}}}(s_t, a_t) $.
\end{theorem}
The proof is in appendix \ref{appx:chap2-soft-policy-update-improvement} for completeness. Both results will be useful in chapter \ref{chapter:chap3} for the analysis of algorithms. Finally, this gives us the basic theoretical finding that is analogous to policy iteration for normal reinforcement learning.

\subsection{Soft Q-Learning}
\label{sec:chap2-soft-q-implementation}
Now we have the closed-form solution for the optimization problem. For implementation procedure, first, let's consider how can we model the value function, from the definition, we can derive the following Monte-Carlo estimation with importance sampling i.e 
\begin{equation}
    V_{\theta}(s_t) = \log \mathbb{E}_{q_{a'}}\Bigg[ \frac{\exp(Q(s_t, a_t)) P_{\prior}(a_t | s_t)}{q_{a'}(a')} \Bigg]
\end{equation}
Now, for the action-value function, we have the following prediction problem with minimizing Mean-square error:
\begin{equation}
\label{eqn:chap2-soft-Q-mse}
    \loss_Q(\theta) = \mathbb{E}_{s_t, a_t}\left[ \frac{1}{2} \left( Q_{\theta}(s_t, a_t) - \hat{Q}_{\theta^-}(s_t, a_t)  \right)^2 \right]
\end{equation}
where the target is defined by
\begin{equation}
\label{eqn:chap2-soft-Q-mse-target}
    \hat{Q}_{\theta^-}(s_t, a_t) = \beta r(s_t, a_t) + \gamma \mathbb{E}_{s_{t+1} \sim p_{s}}\left[ V_{\theta^-}(s_{t+1}) \right]
\end{equation}
Now, for the policy, we start by consider a policy to be in a form of $a_t = f_{\phi}(\xi ; s_t)$ where $\xi \sim \mathcal{N}(0, I)$, which we try to minimize the following function:
\begin{equation}
    \loss(\phi ; s_t) = \kl\left( \pi_{\phi}(\cdot | s_t) \bigg\| \exp\left( Q_{\theta}(s_t, \cdot) - V_{\theta}(s_t)  \right) P_{\prior}(\cdot) \right)
\end{equation}
We can find the direction $\Delta f_{\phi}(\cdot ; s_t)$ that reduces the KL-divergence via SVGD, which gives us following update terms:
\begin{equation}
    \begin{aligned}
        \Delta f_{\phi}(\cdot ; s_t) = \mathbb{E}_{a_t \sim \pi^{\phi}}\bigg[ \kappa \left( a_t, f_{\phi}(\cdot ; s_t) \right) &\nabla_{a'} \bracka{Q_{\theta}(s_t, a') + \log P_\prior(a')} \Big|_{a' = a_t} \\
        &+ \alpha \nabla_{a'} \kappa \left( a', f_{\phi}(\cdot ; s_t) \right) \Big|_{a' = a_t} \bigg]
    \end{aligned}
\end{equation}
Now, the training direction for the policy is:
\begin{equation}
    \frac{\partial \loss(\phi ; s_t)}{\partial \phi} \propto \mathbb{E}_{\xi} \left[ \Delta f_{\phi}(\cdot ; s_t) \frac{\partial f_{\phi}(\xi ; s_t)}{\partial \phi} \right]
\end{equation}

\subsection{Soft Actor-Critic}
\label{sec:chap2-soft-ac-implement}

In Soft Actor-Critic case, we learn 3 networks: Q-function $Q_{\theta} (s_t, a_t)$ and Value-Function $V_{\psi}(s_t)$ and Policy $\pi_{\phi}(a_t | s_t)$. Let's start with value function, instead of using Monte-Carlo estimation, we shall use definition from equation \ref{eqn:chap2-soft-value-def-expand} as the target value, which we can minimize mean-square error:
\begin{equation}
    \loss_V(\psi) = \mathbb{E}_{s_t \sim \mathcal{D}} \left[ \frac{1}{2} \left( V_{\psi}(s_t) - \mathbb{E}_{a_t \sim \pi_{\phi}}\left[ Q_{\theta}(s_t, a_t) - \log \frac{\pi_{\phi}(a_t | s_t)}{P_{\prior}(a_t | s_t)} \right] \right)^2 \right]
\end{equation}
where $\mathcal{D}$ is just the replay memory. The action-value function training objective is the same as soft Q-learning (see equation \ref{eqn:chap2-soft-Q-mse} and \ref{eqn:chap2-soft-Q-mse-target}) but,now, we have independent value function instead. For policy, we directly optimize the KL-divergence, which we can ignore the normalizing terms because it doesn't depend on the action, while the policy can be represented as $f_\phi(\xi ; s_t)$ similar to soft Q-learning:
\begin{equation}
    \loss_{\pi}(\phi) = \mathbb{E}_{s_t \sim \mathcal{D}} \left[ \kl\left( \pi_{\phi}(\cdot | s_t) \bigg| \bigg| \frac{\exp(Q_{\theta}(s_t, \cdot)) P_{\prior}(\cdot | s_t)}{Z_{\theta}(s_t)} \right) \right]
\end{equation}
The authors also use normalizing flow to get the output of the policy to be bounded within a range. We suggested \cite{haarnoja2018softa} for more details.

\section{Introduction to Multi-Agent Problems}
\label{sec:chap2-multi-agent-rl}

\subsection{Problem Formulation and Solution Concepts}
\label{sec:chap2-formulation-concepts}
Now, we have explored single-agent reinforcement learning. Let's move the multi-agent reinforcement learning. We will start by defining the problem definition in a multi-agent case, which is a stochastic game \cite{shapley1953stochastic} formation. 
\begin{definition}
Stochastic Game is a tuple: $\langle S, N, A_1, \cdots, A_N, \mathcal{T}, p_{01}, \cdots, p_{0N}, R_1, \cdots, R_N, \gamma \rangle$ where 
\begin{itemize}
    \item State space: $S = \{s_0, s_1, \cdots , s_{|S|}\}$. 
    \item Number of agents: $N \in \mathbb{N}$. 
    \item Action space for player $i$: $A_i = \{a_0, a_1, \cdots, a_{|A_i|}\}$. 
    \item Transition function: $T : S \times A_1 \times \cdots \times A_N \times S \rightarrow [0, 1]$.  
    \item Initial state distribution for player $i$: $p_{0i}: S \rightarrow [0, 1]$. 
    \item Reward distribution function for player $i$: $r_{i \ t+1} \sim R_i(r_{i \ t+1} | s_{t+1}, a_{1 \ t}, \cdots a_{N \ t})$.
    \item Discount Factor: $\gamma \in \mathbb{R}$. 
\end{itemize}
\end{definition}
\noindent
Now, the goal of agents is to maximize its expected reward
\begin{equation}
    \argmax{\pi_i} \mathbb{E}\brackb{\sum^\infty_{t=0} \gamma ^t r^i(s_t, a^i_t, a^{-i}_t) } 
\end{equation}
This objective might be too vague as we don't know what kind of opponent we are going to play against, and the same problem also applied to the opponent. Let's simplify the problem by \textit{fixing} the opponent and then updating the agent's policy alone. By doing this, we reduce the problem into a single-agent problem, which we can define this as the agent that best responses to a particular set of opponent i.e 
\begin{equation}
\begin{aligned}
    \operatorname{BR}^i(\pi_1, \cdots, &\pi_{i-1}, \pi_{i+1}, \cdots, \pi_N) \\
    &= \argmax{\pi_i} \mathbb{E}_{\pi_1, \cdots , \pi_i, \cdots, \pi_N}\brackb{\sum^\infty_{t=0} \gamma ^t r^i(s_t, a^1_t, \cdots, a^i_t, \cdots, a^{N}_t) }
\end{aligned}
\end{equation}
Now, Nash equilibrium is the set of polices\footnote{Or sets of policies as a game can have more than 1 Nash equilibrium} where every policy doesn't benefit from any deviation, suppose $\pi_1^*, \cdots, \pi_N^*$ are set of policies that are Nash equilibrium then
\begin{equation}
\begin{aligned}
    \forall i \in [1, N], \forall \pi_i \in \Pi_i : \mathbb{E}_{\pi_1^*, \cdots, \pi_i^*, \cdots \pi_N^*} &\brackb{\sum^\infty_{t=0} \gamma^t  r^i(s_t, a^1_t, \cdots, a^i_t, \cdots, a^{N}_t) } \\
    &\ge \mathbb{E}_{\pi_1^*, \cdots, \pi_i, \cdots \pi_N^*} \brackb{\sum^\infty_{t=0} \gamma^t  r^i(s_t, a^1_t, \cdots, a^i_t, \cdots, a^{N}_t) }
\end{aligned}
\end{equation}
Nash equilibrium exists for all the games \cite{nash1950equilibrium}, which is a very important result. Nash equilibrium is where we should aim, however, solving Nash equilibrium is in complexity class PPAD, which is a subset of NP problems \cite{daskalakis2009complexity} and even approximating it is PPAD-complete \cite{daskalakis2013complexity}. We will leave the problem of define the \textit{optimality} of agent here and will explore it in chapter \ref{chapter:chap3}. Finally, we would like to denote the action that are not coming from agent $i$ as $a^{-i}$ i.e $a^{-i} = (a^1, \cdots, a^{i-1}, a^{i+1}, \cdots, a^N)$. Finally, we define cooperative game to be the game that all the agents are optimizing a common reward.

\subsection{Classical Reinforcement Algorithms}
We will consider two basic algorithms for solving multi-agent reinforcement learning, which are both value function based\footnote{We find the optimal Q-value and then acts according to it.}. Starting with Nash Q-learning \cite{hu2003nash}, which we aim to estimate Nash Q-function. It is defined as: Given a set of Nash Equilibrium policies $\pi_1^*, \cdots, \pi_N^*$, we have 
\begin{equation}
    Q^i_{\operatorname{Nash}}(s_t, a^1_t, \cdots, a^N_t) = r(s_t, a^1_t, \cdots, a^i_t, \cdots, a^N_t) + \gamma\mathbb{E}_{s_{t+1}}\brackb{V^i(s_{t+1}, \pi_1^*, \cdots, \pi_N^*)}
\end{equation}
where $V^i$ is the cumulative discounted reward that follows $\pi_1^*, \cdots, \pi_N^*$. Based on the use of maximum to estimate optimality Q-value\footnote{See equation \ref{eqn:chap2-optim-Q-and-V} and \ref{eqn:chap2-Q-stochastic-update}} we can, instead, use the following operation:
\begin{equation}
    \operatorname{Nash} Q^i_t(s) = \pi^1(s)\cdots\pi^n(s)Q^i(s, a_1, \cdots, a_n)
\end{equation}
which is the expected payoff of Nash equilibrium solution to matrix game\footnote{There is one state game with immediate reward given once all the actions are executed.} given a list of value functions $Q^1(s', \vec{a}), \cdots, Q^N(s', \vec{a})$ as a payoff function. The authors shown the following operator 
\begin{equation}
    \contractop Q^i(s', a_1, \cdots, a_N) = r_i + \gamma \mathbb{E}_{s_{t+1}}\brackb{\operatorname{Nash} Q^i_t(s_{t+1}) }
\end{equation}
is a contraction mapping, which means when repeated applying the map will lead to Nash action value function. The problem with Nash-Q learning is how can we select that Nash equilibrium. \cite{littman2001friend} proposed to use friend-or-foe Q-learning that is based on the identification of the agents whether they are friends (i.e trying to maximize our reward) or foes (i.e trying to minimize our reward), and using this to update the value,
\begin{equation}
    \max_{\pi_1, \cdots, \pi_k} \min_{\pi_k, \cdots, \pi_N}\sum_{s'} \pi^i(s')\cdots\pi^n(s')Q^i(s, a_1, \cdots, a_n)
\end{equation}
Friend-or-foe Q-learning has the same convergence property as Nash Q-learning\footnote{\textit{Spolier}: Balacing Q-learning is a \correctquote{soft} version of friend-or-foe Q-learning}, but has a unique solution unlike the Nash equilibrium solution.

\section{Deep Multi-Agent Reinforcement Learning}
\label{sec:chap2-deep-marl}

\subsection{Policy Based}

\subsubsection{Counterfactual Multi-Agent Policy Gradients (COMA-PG)}
We will consider solving cooperative multi-agent problems in which the main problem with this setting is the fact that it is hard for the training algorithm to assign correct credit to correct agent. COMA-PG \cite{foerster2018counterfactual} solved this problem by proposing a following baseline for a policy gradient algorithm:
\begin{equation}
    D^i = r(s, a) - r(s, (a_c^i, a^{-i}))
\end{equation}
where $a_c$ is the \textit{default action} for agent $i$. The default action can be, simply, an expected action, so now the advantage function becomes:
\begin{equation}
    A^i(s, a) = Q(s, a) - \sum_{a'_i} \pi^i(a'_i | s) Q(s, (a'_i,a^{-i}))
\end{equation}
which we can use it to train a normal policy gradient algorithm.

\subsubsection{Multi-Agent DDPG (MADDPG) \cite{lowe2017multi}}
Now, we can consider the policy gradient for any multi-agent problem as:
\begin{equation}
\nabla_{\theta_i}J(\theta_i) = \mathbb{E}_{s\sim d(s), a_i \sim \pi_i} \left[ \nabla_{\theta_i} \log \pi_i(a_t | s_i) Q^{\pi}_i(s, a_1, \dots, a_N) \right]
\end{equation}
Now, they extended this to deterministic policy (similar to DDPG) $\mu_{\theta_i}$, which is 
\begin{equation}
    \nabla_{\theta_i}J(\mu_{\theta_i}) = \mathbb{E}_{s, a \sim \mathcal{D}} \left[ \nabla_{\theta_i} \mu_{\theta_i}(a_i | s_i) \nabla_{a_i} Q^{\mu}_i(s, a_1, \dots, a_N) |_{a_i = \mu(s_i)} \right]
\end{equation}
where $\mathcal{D}$ is experience replay that contains a tuple $(s, s', a_1, \dots, a_N, r_1, \dots, r_N)$, and, finally, critic can be trained by minimizing the following objective 
\begin{equation}
    \mathcal{L}(\theta_i) = \mathbb{E}_{(s, a, s', r) \sim \mathcal{D}}\left[ \left( Q^{\mu}_i (s, a_1, \dots, a_N) - r_i - \gamma Q^{\bar{\mu}}_i(s', a'_1, \dots, a'_N) |_{a'_i = \bar{\mu}(s_i')} \right)^2  \right]
\end{equation}
There is a notion of \textit{centralized} as we have to have a full knowledge of other agent's policy, while \textit{decentralized}\footnote{This is what we aim for} is where we doesn't assume any knowledge of other agent's inner policy, but we can still observe their actions. 

\subsection{Value Function Based}
Now we have see the use of value estimation in MADDPG, let's consider the use of centralized value function estimation. Please note that all the algorithms in this section assume cooperative setting. The main intuition behind all value function based is that the action value function Q can be decomposed to each individual agent's value. In value decomposition network (VDN) \cite{sunehag2017value}, the author assume that:
\begin{equation}
\begin{aligned}
Q(s (a^1, \dots, a^N)) \approx \sum^N_{i=1} \tilde{Q}_i(s, a^i)
\end{aligned} 
\end{equation}
Note that this works nicely in the case of cooperative setting as in competitive game the value function mixing is likely to be more \textit{complex} interactions, in contrast to, cooperative as they are likely moving toward the same goal. QMIX \cite{rashid2018qmix} is an extended version of VDN, by including a mixing network, instead of sum. However, there has to be some constrain on the function. Notably, 
\begin{equation}
\underset{a}{\arg\max} \ Q_{\text{tot}}(s, a) \ = \begin{pmatrix}
	\underset{a}{\arg\max} \ Q_1(s, a) \\ \underset{a}{\arg\max} \ Q_2(s, a) \\ \vdots \\ \underset{a}{\arg\max} \ Q_N(s, a) \\
\end{pmatrix}
\end{equation}
The VDN satisfies the condition but there is a more generalized notion of this, which is monotonicity. 
\begin{equation}
\forall n \in [N] : \frac{\partial Q_{\text{tot}}}{\partial Q_{n}} \ge 0
\end{equation}
This can be done by making the mixing function positive ($\ge 0$), which is proved to be a good approximation of monotonic function, which can be easily implemented using ReLU function \cite{nair2010rectified}. The training is standard Q-value objective. Finally, as shown in \cite{mahajan2019maven}, QMIX can suffers from learning suboptimal action value function and poor exploration. To prevent this from happening, MAVEN \cite{mahajan2019maven} learns multiple modes of value functions given latent variable $z$
\begin{itemize}
    \item The value function is conditioned on shared latent variables $z$ which is controlled by hierarchical policy $z = g_\theta(x, s_0)$ where $x$ is value that is sampled from simple distribution $P(x)$
    \item For $z$, each joint action-value function is monotonic approximation to optimal action value function $g_{\eta}^a(o^a_{1:t}, a_{1:t}, u^{a}_{1:t-1})$ which also consists of $g_{\phi}(z, a)$ (Hyper-net as to modify the utility for particular mode of exploration). Now all value function are mixed using $g_{\psi}$ function 
\end{itemize}
We will train the Q-network as following 
\begin{equation}
    \mathcal{L}_{Q}(\phi, \eta,\psi) = \mathbb{E}_{\pi_a}  \left[ Q(u_t, s_t ; z) - \left[r(u_t, s) + \gamma \max_{u_{t+1}} Q(u_{t+1}, s_{t+1} ; z)\right] \right]
\end{equation}
while the hierarchical policy is trained cumulative rewards $R(\tau, z | \phi, \eta, \psi)= \sum_t r_t$ 
\begin{equation}
    \mathcal{L}_{RL}(\theta) = \int R(\tau_A | z)P_{\theta}(z | s_0) \ \dby z\dby s_0
\end{equation}
To prevent collapse, mutual information objective for a trajectory $\tau = \{(u_t, s_t)\}$ is introduced where we need $\sigma$ which returns Boltzman policy based on utility, while the trajectory encode via RNN. Now the mutual information loss is lower bounded by
\begin{equation}
\begin{aligned}
    \mathcal{L}_{ML} &= \mathcal{H}(\sigma(\tau)) - \mathcal{H}(\sigma(\tau) | z) = \mathcal{H}(z) - \mathcal{H}(z | \sigma(z)) \\
    &\ge \underbrace{\mathcal{H}(z) + \mathbb{E}_{\sigma(z), z} \left[ \log q_v(z | \sigma(\tau)) \right]}_{\mathcal{L}_v(\phi, \eta, \psi, v)}
\end{aligned}
\end{equation}
Bad variational approximation can hurt the performance as induces gap. We can set the auxiliary reward as $r^z_{\text{aux}} (z) = \log(q_v(z | \sigma(z))) - \log  (p(z))$
The final objective is 
\begin{equation}
    \max_{v, \phi, \eta, \psi, \theta} \mathcal{L}_{RL}(\theta) + \lambda_{MI}\mathcal{L}_v(\phi, \eta, \psi, v) - \lambda_{Q}\mathcal{L}_{Q}(\phi, \eta,\psi)
\end{equation}

\section{Other Related Algorithms}
\label{sec:chap2-other-algos}
This is the list of the algorithms that will be introduced in the following chapters: 
\begin{itemize}
    \item \hyperref[chapter:chap3]{Chapter 3}: Regularized Opponent Model with Maximum Entropy Objective (ROMMEO) \cite{tian2019regularized}, Probabilistic Recursive Reasoning (PR2) \cite{wen2019probabilistic}, and Balancing Q-learning \cite{grau2018balancing}.
    % \item \hyperref[chapter:chap5]{Chapter 5}\footnote{\hyperref[chapter:chap4]{Chapter 4} doesn't require any additional introduce but we assuming the readers have read \hyperref[chapter:chap3]{Chapter 3}}:  Maximum a Posteriori Policy Optimization (MPO) \cite{abdolmaleki2018maximum}, VIREL \cite{fellows2019virel}, and its on-policy variance V-MPO \cite{song2019v}. 
    \item \hyperref[chapter:chap6]{Chapter 6}: Soft Hierarchical reinforcement learning \cite{igl2019multitask, lobo2019soft} and Delayed communication or Dynamic policy termination Q-learning \cite{han2019multi}.
    % \item \hyperref[chapter:chap7]{Chapter 7}: Inverse reinforcement learning starting with Generative Adversarial Imitation Learning (GAIL) \cite{ho2016generative}, Adverserial Inverse Reinforcement Learning (AIRL) \cite{fu2017learning} with its connection between Generative Adversarial Network (GAN) \cite{goodfellow2014generative, finn2016connection}. Furthermore, we have unified view of adversarial imitation learning via $f$-GAN \cite{nowozin2016f} formulation \cite{ghasemipour2019divergence, ke2019imitation}. Also, we will also consider context awareness reinforcement algorithm \cite{yu2019meta, rakelly2019efficient}, and finally, multi-agent imitation learning including Multi-Agent GAIL (MAGAIL) \cite{song2018multi}, Multi-agent Adversarial Inverse Reinforcement Learning (MA-AIRL) \cite{yu2019multi} and, Decentralized Adversarial Imitation Learning algorithm with Correlated policies \cite{liu2020multi}.Last but not least the probabilistic interpretation of GAN, which includes Bayesian Generative Adversarial Imitation Learning \cite{jeon2018bayesian}, Importance weight GAN (IWGAN) \cite{hu2017unifying}, and $\alpha$-GAN \cite{rosca2017variational}, which is an extension of Adversarial autoencoders (AAE) \cite{makhzani2015adversarial}\footnote{As mentioned before, this chapter will be part-survey and part proposal to the new multi-agent imitation learning}.
    % \item \hyperref[chapter:chap8]{Chapter 8}: Generalized Recursive Reasoning (GR2) \cite{wen2019multi}
    \item \hyperref[chapter:chap9]{Chapter 9}: Variational Sequential Monte Carlo, Deep Variational Reinforcement Learning \cite{igl2018deep, shvechikovjoint} and Sequential Variational Soft Q-Learning (SVQN) \cite{huangsvqn}. They are the basis for representing belief, which is a basis for public belief MDP \cite{nayyar2013decentralized} and Bayesian Action Decoder (BAD) \cite{foerster2018bayesian}. 
\end{itemize}

% \section{Extensions to control as inference frameworks}
% After we have introduced control and inference framework, we will now present current progress of reinforcement learning made based on this framework. We will start with hierarchical agent, which is simplest extension to the framework. After this, we will look on how can be represents the unknown environment with belief over the state, which would requires us to formulate the problem into partially observable Markov decision process (POMDP). Finally, we will take a slight turn and examine current work on information theoretic inspired single agent reinforcement learning algorithms.

% \section{Multi-agent reinforcement learning}
% In this section, we will start formulating Multi-agent reinforcement learning problems, by first introducing a formulation of decentralized partially observable Markov decision process (DEC-POMDP), introducing some game theoretic concepts, with more traditional algorithms including Nash-Q \cite{hu2003nash}, MADDPG \cite{lowe2017multi}, COMA \cite{foerster2018counterfactual}, where some of them will be our baseline for our algorithms.

% \section{Multi-agent with probabilistic inference framework}
% Finally, we will present current algorithms developed by re-interpretation multi-agent reinforcement learning problem as control-as-inference framework. We will present 4 algorithms based on this: PR2 \cite{wen2019probabilistic}, ROMMEO \cite{tian2019regularized}, GR2 \cite{wen2019multi}, and Balancing-Q \cite{grau2018balancing}. All of them will be examined in full details because they lays a foundations to main result of this thesis. We will end this chapter by shinning light into incompatibility between these algorithms, which will solve in the later chapters. 