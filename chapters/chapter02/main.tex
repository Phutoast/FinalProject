\label{chapter:chap2}
\begin{adjustwidth}{1cm}{}
    % In this chapter, we will provide and reviews tools that will be useful for deriving probabilistic multi-agent reinforcement learning algorithms with a formal setting of the problem that the algorithms are trying to solve. We will start by going through the probabilistic machine learning, which provides the tools for mitigate an inevitable intractable problem (will discuss in more details). After getting all the fundamental tools, we will move to single agent reinforcement learning section, which, in the first part, will provide classical results, and theorem. In the later section, we will investigate control-as-inference framework and its extensions. Finally, we will finish with an introduction to multi-agent reinforcement learning with Multi-agent with probabilistic inference framework (MAPI) - our main focus. Our main focus for this chapter is to provide readers backgrounds of the current sub-field, and how they are developed, which will help them understand the motivation for our result more clearly.
    The main aim of this chapter is to make the readers familiar with the background required to understand the development of the algorithms proposed in this thesis. We try to keep this chapter as small as possible. The details of the structure of this chapter will be provided in section \ref{sec:chap1-Outline}. As noted before, and as the way to make this chapter compact, any specific algorithms that will only be discuss in specific chapter will not be describe here; this will also help finding related information easier. At the end of this chapter, in section \ref{sec:chap2-other-algos}, we will list the algorithms that will be introduce given the corresponding chapters.
\end{adjustwidth}


\section{Probabilistic Method in Machine Learning}
\label{sec:chap2-prob-ml}

\subsection{Introduction To Baysian Method}
\label{sec:chap2-intro-baysian}
Most of the focus in this section will be to provide techniques that provides an approximate solution to Bayes' rule, suppose we want to infer a parameter $\theta$ given training data $X$ and training label $Y$ in supervised learning: 
\begin{equation}
    \label{eqn:chap2-bayes-theorem}
    P(\theta | X, Y) = \frac{P(Y | X, \theta) P(\theta)}{P(Y | X)} = \frac{P(Y | X, \theta) P(\theta)}{\int P(Y | X, \theta) P(\theta ) \dby \theta}
\end{equation}
For the prediction task, given the input data point $x^*$ that we would like to make prediction $y^*$ of, we can do marginalization of all parameters i.e
\begin{equation}
    \label{eqn:chap2-bayes-pred}
    \int P(y^* | x^*, X, Y, \theta) \dby\theta = \int P(y^* | x^*, \theta) P(\theta | X, Y) \dby\theta
\end{equation}
The biggest problem in Bayesian inference is when the integral is intractable especially in the denominator of second equality in equation \ref{eqn:chap2-bayes-theorem}. Note that in equation \ref{eqn:chap2-bayes-pred}, we can approximate it via Monte-Carlo sample assuming we can sample $P(\theta | X, Y)$. In this thesis, we will focus on variational inference techniques, however, there are other techniques for example, conjugate prior.
% \Phu{Do i have to cite Learning in Graphical Models ?}

\subsection{Variational inference techniques}
\label{sec:chap2-vi-technique}
We would like to approximate the posterior $P(\theta | X, Y)$ with parametric distribution $q_{\phi} (\theta)$. This can be done by minimizing Kullback–Leibler divergence(KL-divergence)\footnote{Note that given 2 difference distributions $q$ and $p$, $\kl(q \| p) \ne \kl(p \| q)$.} between them:
\begin{equation}
    \kl\left( q_{\phi} (\theta) \Big\| P(\theta | X, Y) \right) = \int q_{\phi} (\theta) \log\left( \frac{q_{\phi} (\theta)}{P(\theta | X, Y)} \right) \dby \theta
\end{equation}
\textbf{Minimizing} the KL-divergence is equivalent to \textbf{maximized} the following value, which we will called it Evidence lower bound (ELBO) defined by
\begin{equation}
\begin{aligned}
    \argmax{\theta} \mathbb{E}_{q_{\phi} (\theta)} \left[ P(Y | X, \theta) \right] - \kl \left(q_{\phi} (\theta) \Big\| P(\theta) \right) &= \argmax{\theta} \int q_{\phi}(\theta)\log\left(\frac{P(Y, \theta | X)}{q_{\phi}(\theta)}\right) \dby \theta \\
    &= \argmin{\theta } \kl\left( q_{\phi} (\theta) \Big\| P(\theta | X, Y) \right)
\end{aligned}
\end{equation}
Given this can calculate this without relying on $P(Y | X)$, the intractable terms. The full derivation is in appendix \ref{appx:chap2-elbo-kl} for completeness. Furthermore, we can derive ELBO based on Jensen's inequality, which is shown in \ref{appx:chap2-elbo-jensen}.\footnote{We think that minimizing the KL-divergence makes more sense compared to Jensen's inequality. However, we still have to mention it because some of the key papers ultized this technique.}

% Now, we will step back and define the log-likelihood, note that we can also derive ELBO from this quantity using Jensen's inequality, see appendix \ref{appendix-1:elbo-jensen}
% \begin{equation}
%     l(X, Y) = \log P(Y| X) = \log \left( \int P(Y,  \theta | X) \dby \theta \right)
% \end{equation}
% % We will treat $\theta$ as hidden variable, and then finding how can we increase the likelihood of this, which will provides the reason behind the maximizing ELBO. 
% We can show the gap between this log-likelihood that we want to optimize and ELBO 
% \begin{equation}
%     l(X, Y) - \underbrace{\int q_{\phi}(\theta) \log\left(\frac{P(Y, \theta | X)}{q_{\phi}(\theta)}\right) \dby \theta}_{\text{ELBO}} = \int q_{\phi} (\theta) \log\left( \frac{q_{\phi} (\theta)}{P(\theta | X, Y)} \right) \dby \theta
% \end{equation}
% The derivation will be in appendix \ref{appendix-1:elbo-gap}. Interestingly, it is the same as the objective that we want to optimize earlier, which is KL-divergence between variational distribution and true posterior. We would like to note that minimizing this is possible, since it is shown earlier to be the same as maximizing ELBO but it is intractable to calculate its value. Finally, KL-divergence is always non-negative (this is known as Gibbs' inequality), therefore ELBO will stay lower bound, and optimizing ELBO will also optimize whole $l(X, Y)$

\subsection{Expectation Maximization (EM) Algorithm}
\label{sec:chap2-em-algo}
Now after we have explored the use of variational inference technique, we will now turn to EM algorithm. We will follows the example provided in \cite{fellows2019virel}. Suppose, we have a $N$ data points $X = \{x^{(n)}\}^N_{n=1}$ that we know that it is generated by each hidden variable $z = \brackc{z^{(n)}}^N_{n=1}$. We would like to learn the "process" that generates such a data point, i.e the variable $\theta$, such that $P_{\theta}(X | z)$. It is clear that we would like to find $\theta$ that maximizes the log-likelihood of the observed data point
\begin{equation}
    l_{\text{unsup}}(\theta) = \log \left( P_{\theta}(X) \right) = \log\bracka {\int P_{\theta}(X, z) \dby z}
\end{equation}
With this, we can also introduce the variational distribution over the posterior hidden variable (since we don't know what $P(z | X)$ is) $q_{\phi}(z)$, with similar pattern, we can decompose the marginal log-likelihood into:
\begin{equation}
    \begin{aligned}
        l_{\text{unsup}}(\theta) &= \int q_{\phi}(z) \log \bracka{\frac{P(X, z)}{P(z | X)}} \dby z \\
        &= \underbrace{\int q_{\phi}(z) \log \bracka{\frac{P_{\theta}(X, z)}{q_{\phi}(z)}} \dby z}_{\circled{1}} + \underbrace{\int q_{\phi}(z) \log \bracka{\frac{q_{\phi}(z)}{P_{\theta}(z | X)}} \dby z}_{\circled{2}}
    \end{aligned}
\end{equation}
We can show the equality above is true in \ref{appx:chap2-elbo-gap}. Now, if we look carefully we can see that part $\circled{2}$ is KL-divergence between $q_{\phi}(z)$ and $P_{\theta}(z | X)$. In theory, we can have the exact version of EM algorithm as follows: at step $k$, we first make the gap between part $\circled{1}$ and log-likelihood tight by setting the exact posterior to the variational distribution $q_{\phi}(z)$
\begin{equation}
    q_{\phi^{(k+1)}}(z) \leftarrow P_{\theta^{(k)}}(z | X)
\end{equation}
We call this \textit{Expectation-Step} or \textit{E-step}, then we can optimizing $\theta$ by maximizing the part $\circled{2}$ of the log-likelhood:
\begin{equation}
    \theta^{(k+1)} \leftarrow \argmax{\theta^{(k)}}  \int P_{\theta^{(k)}}(z | X) \log \bracka{\frac{P_{\theta^{(k)}}(X, z)}{P_{\theta^{(k)}}(z | X)}} \dby z 
\end{equation}
We can this \textit{Maximization-Step} or \textit{M-Step}. However, it is clear that in M-step we are maximizing ELBO with respect to $\theta$, while in E-step we maximizing ELBO with respect to $\phi$\footnote{Recall from section \ref{sec:chap2-vi-technique} that minimizing KL is same as maximizing ELBO}. Now, with this \correctquote{coincident}, we can perform variational EM\footnote{We are using variational distribution to approximate the posterior.} based on the following ELBO:
\begin{equation}
    \begin{aligned}
        \mathcal{L}(\theta, \phi) &= \int q_{\phi}(z) \log \bracka{\frac{P_{\theta}(X, z)}{q_{\phi}(z)}} \dby z
        = \mathbb{E}_{q_{\phi}(z)} \brackb{\log P_{\theta}(X | z)} - \kl\bracka{q_{\phi}(z) \Big\| P(z)}
    \end{aligned}
\end{equation}
Now, the training scheme for variational EM is as follows (at step $k$):
\begin{equation}
    \begin{aligned}
        &\text{E-Step:} \quad \phi^{(k+1)} \leftarrow \argmax{\phi^{(k)}}\mathcal{L}(\theta^{(k)}, \phi^{(k)}) \qquad \text{M-Step:} \quad \theta^{(k+1)} \leftarrow \argmax{\theta^{(k)}}\mathcal{L}(\theta^{(k)}, \phi^{(k+1)})
    \end{aligned}
\end{equation}

\subsection{Stein Variational Gradient Descent (SVGD) \cite{liu2016stein}}
SVDG is used in many of the algorithm that we are presenting (for example soft Q-learning \cite{haarnoja2017reinforcement} and PR2 \cite{wen2019probabilistic}). It is a \correctquote{general purpose variational inference algorithm} \cite{liu2016stein}, which can also be used to "sample" the data point from complex distribution (optimized variational distribution). This is extremely useful since we want to construct and use neural network based on variational distribution for our agents. 

%\grammar{The algorithm itself is simple, yet, having very interesting mathematical formulation behind it. We suggest interested reader to check out the paper. We will described the general form of the algorithm, while the details will be left for each of its usage.} 

Suppose that we want to sample the distribution\footnote{which is usually unnormalized i.e the denominator, similar to \ref{eqn:chap2-bayes-theorem} is intractable} $P$ by a set of $n$ particles $\{\xi^{(0)}_i\}^n_{i=1}$. Then we can update the set of such particles such that the final set of $n$ particles $\{\xi^{(T)}_i\}^n_{i=1}$ represent sampled point from the distribution $P$. The algorithm does the following update rule (for iteration at step $t$): 
\begin{equation}
    x^{(t+1)}_i \leftarrow x^{(t)}_i + \alpha \bracka{\frac{1}{n}\sum^n_{j=1} \brackb{k(x^{(t)}_j, x^{(t)}_i) \nabla_{x^{(t)_j}} \log P(x^{(t)}_j) + \nabla_{x^{(t)_j}} k (x^{(t)}_j, x^{(t)}_i) }}
\end{equation}
where $\alpha$ is the updating step and we can use other optimization methods, such as ADAM \cite{kingma2014adam}, to optimize the points. 

% \factcheck{The author has shown that this update rule follows the functional gradient of KL-divergence between the variational distribution ($q_{[T]}$ which is the density of $z = T(\xi)$) and the true distribution $P$ (Theorem 3.3). See the original paper for more details.}

% \grammar{This is used in variational auto-encoder \cite{kingma2013auto} where by we parameterized encoder as $q_{\phi}(z | X) = \mathcal{N}(z ; \mu_{\phi}(X), \Sigma_{\phi}(X))$, where $\mu_{\phi}$ and $\Sigma_{\phi}$ are neural network. For, the decoder as $P_{\theta}(X | z) = \mathcal{N}(X ; \mu_{\theta}(z), \Sigma_{\theta}(z))$. However, for decoder we usually ignore the covariance matrix, and for encoder, we tends to assume independences between each latent variables (diagonal covariance matrix) for tractable computation. Since we have both parameters available, we can train both neural networks at the same time by optimizing the ELBO.}
% \begin{equation}
%     \mathbb{E}_{q_{\phi}(z | X)} \brackb{\log P_{\theta}(X | z)} - \kl\bracka{q_{\phi}(z | X) \Big\| P(z)}
% \end{equation}
% \grammar{The authors used re-parametrisation trick (pathwise derivative \cite{gal2016uncertainty}) in order to optimize the first quantity in ELBO. Generally, suppose we want to find the gradient of $\mathbb{E}_{a \sim p_{\theta}}[f(a)]$. This can be done by finding differentiable function $g(\theta, \varepsilon)$, where $\varepsilon \sim \mathcal{X}$ and $\mathcal{X}$ is arbitrary distribution that we can sample, such that $\mathbb{E}_{\varepsilon \sim \mathcal{X}}\brackb{f(g(\theta, \varepsilon))} = \mathbb{E}_{a \sim p_{\theta}}[f(a)]$. For example, for normal distribution $\mathcal{N}(x; \mu, \sigma)$, we have $g(\mu, \sigma, \varepsilon) = \mu +\sigma\varepsilon$ where $\varepsilon \sim \mathcal{N}(\varepsilon;0, 1)$. The gradient, thus, can easily be computed by chain rule.}

\section{Single Agent Reinforcement Learning}
\label{sec:chap2-single-rl}
In this section, we will go through the classical notion of single agent reinforcement learning as a basis to the next section, which is about deep reinforcement learning - a scale and approximated version of algorithms discuss here.

\Phu{citing the courses and book ?}
\subsection{Markov Decision Process}
\label{sec:chap2-rl-defintions}
Markov Decision process (MDP) formulates the task that reinforcement learning trying to solve\footnote{It is a mathematical description of \correctquote{game}.}. 
\begin{definition}
    Markov decision process (MDP) is a tuple : $\langle S, A, R, T, p_0, \gamma \rangle$ where 
    \begin{itemize}
        \item State space: $S = \{s_0, s_1, \cdots , s_{|S|}\}$. Usually we have $s_i \in \mathbb{R}^{d_s}$
        \item Action space: $A = \{a_0, a_1, \cdots a_{|A|}\}$
        \item Reward function: $\mathcal{R}: S \times A \rightarrow \mathbb{R}$
        \item Transition function: $T: S \times A \times S \rightarrow [0, 1]$, where $s^{(t+1)} \sim T(s^{(t+1)} | s^{(t)}, a^{(t)})$ 
        \item Initial state distribution: $p_0 : S \rightarrow [0, 1] $, where $s^{(0)} \sim p_0$
        \item Discounted factor: $\gamma \in (0, 1]$
    \end{itemize}
    In almost all the cases, we include player's policy $\pi: S \rightarrow \Delta(A)$ where $\Delta(A)$ is probability simplex over $A$, where $a^{(t)} \sim \pi(a^{(t)} | s^{(t)})$ denote player choose action $a_t$ at state $s_t$ at time $t$. The goal of any reinforcement learning algorithms is to maximizes the reward with or without full description/experience of MDP.
\end{definition}
\noindent
We denote the value function of the player's policy $V^\pi(s)$ as the expected future cumulative reward given the state the agent is in:
\begin{equation}
    V^\pi(s) = \mathbb{E}_{a_t \sim \pi, s_t \sim T}\brackb{\sum^\infty_{t=0} \gamma^t r(s_t, a_t) \Bigg| s_0 = s}
\end{equation}
And the action value function is defined as which is the expected future reward when agent perform action $a$ at state $s$: 
\begin{equation}
    Q^\pi(s, a) = \mathbb{E}_{a_t \sim \pi, s_t \sim T}\brackb{\sum^\infty_{t=0} \gamma^t r(s_t, a_t) \Bigg| s_0 = s, a_0 = a}
\end{equation}
The advantage function of an agent is simply the differences between value function and action value function i.e $A^{\pi}(s, a) = Q^\pi(s, a) - V^\pi(s)$. Furthermore, We can establish the connection between both $V(s)$ and $Q(s, a)$ functions as
\begin{equation}
    \label{eqn:chap2-Q-and-V}
    Q^\pi(s, a) = r(s, a) + \gamma \mathbb{E}_{s' \sim T(s' | s, a)}\brackb{V^\pi(s')} \quad \quad \quad V^\pi(s) = \mathbb{E}_{a \sim \pi(a | s)} \brackb{Q^\pi(s, a)}
\end{equation}
We denote the optimal value function and optimal action value function as $V^*(s)$ and $Q^*(s, a)$ when $V^{\pi^*}(s)$ and $Q^{\pi^*}(s, a)$, respectively, where $\pi^* \in \arg\max_{\pi} V^{\pi}(s)$ for any state $s$. Now, we want to establish connection between these 2 optimal functions, which is:
\begin{equation}
\label{eqn:chap2-optim-Q-and-V}
    Q^*(s, a) = r(s, a) + \gamma \mathbb{E}_{s' \sim T(s' | s, a)}\brackb{V^*(s')} \quad \quad \quad V^*(s) =\max_{a\in A} Q^*(s, a)
\end{equation}
We can now slightly change the objective of reinforcement learning to \correctquote{How to find such an optimal policy}.

\subsection{Policy Iteration}
\label{sec:chap2-policy-iter}
We will now introduce the first algorithm that will solve the problem of reinforcement learning, called policy iteration. The strategy is based on simple observation: the agent will always doing better or equal if we set it's behavior to be following the optimal value function in equation \ref{eqn:chap2-optim-Q-and-V} i.e we can improve our policy $\pi$ by greedily chooses the action that would maximized expected future rewards given $Q^\pi$:
\begin{equation}
    \label{eqn:chap2-greedy-Q}
    \pi'(a | s) = \begin{cases}
        1 &\text{ if } a = \argmax{a \in A} Q^\pi(s, a) \\
        0 &\text{ otherwise }
    \end{cases}
\end{equation}
we can call this step, a policy improvement step. One can show that $\forall s \in S: V^\pi(s) \le V^{\pi'}(s)$. Furthermore, we can show that we will reach optimal value function if we keep improving the policy. The proof, for completeness, will be presented in appendix \ref{appx:chap2-rl-policy-improve}. What we have left is to find an algorithm that gives us value function $V^{\pi}(s)$ for any policy $\pi$, which we will call this step \textit{policy evaluation}. We can expanded the equality in equation \ref{eqn:chap2-Q-and-V} as:
\begin{equation}
    V^\pi(s) = \mathbb{E}_{a \sim \pi(a | s)} \brackb{r(s, a) + \gamma \mathbb{E}_{s' \sim T(s' | s, a)}\brackb{V^\pi(s')}} 
\end{equation}
We call this equation expected Bellman equation along with the following expected Bellman operator $\contractop^{\pi} : \mathbb{R}^{|S|} \rightarrow \mathbb{R}^{|S|}$ defined as:
\begin{equation}
    \label{eqn:chap2-exp-bellman-operator}
    \contractop^{\pi} V(s) = \mathbb{E}_{a \sim \pi(a | s)} \brackb{r(s, a) + \gamma \mathbb{E}_{s' \sim \contractop(s' | s, a)}\brackb{V(s')}} 
\end{equation}
To solve the policy evaluation problem, we want to find $V(s)$ such that $\contractop^\pi V(s) = V(s)$ however, to solve this via matrix form would be too expensive. However, turn out this expected Bellman operator is an contraction mapping on $\infty$-norm i.e 
\begin{equation*}
    \| \contractop^\pi V_1(s) - \contractop^\pi V_2(s) \|_\infty \le \alpha \|V_1(s) - V_2(s)\|_\infty
\end{equation*}
where $\|V(s)\|_\infty = \max_{s \in S} V(s)$. For some $0 \le \alpha < 1$ (See appendix \ref{appx:chap2-rl-expected-bell-contract} for proof), then by Banach fixed-point theorem
\begin{theorem}{(Banach fixed-point theorem \cite{murfet_2019})}
    Given the complete (every Cauchy sequence converges to a point in that space) metric space $(X, d(\cdot, \cdot))$ and the contraction mapping $\contractop : X \rightarrow X$, the sequence $(\contractop^{(n)} x)^{\infty}_{n=1}$ converges to a fixed point $x^*$ where $\contractop x^* = x^*$, for all points $x \in X$. 
\end{theorem}
repeatedly apply the Bellman operator will lead us to a solution of $V^{\pi}$.  In conclusion, Policy iteration is an algorithm that solves the MDP by involving the alternation between the following 2 steps
\begin{enumerate}
    \item \textbf{Policy Evaluation}: Starting with randomized value function and Repeatedly applying Bellman operator defined in equation \ref{eqn:chap2-exp-bellman-operator} until converge to get true value function $V^{\pi}(s)$.
    \item \textbf{Policy Improvement}: Update the policy by choosing the best action from action value function (we can calculate this by the equation \ref{eqn:chap2-Q-and-V}) following the equation \ref{eqn:chap2-greedy-Q}.
\end{enumerate}
Since the value function always increases in every state, this algorithm is guaranteed to reach the optimal value function and policy. 

\subsection{Value Iteration}
\label{sec:chap2-value-iter}
Now, it is quite computational ineffective to evaluates the policy every times to update the policy. Can we just update the value function using single Bellman update and then uses the policy improvement to come-up with the next value function? The answer is \textit{Yes}. We define optimal value Bellman update operator as:
\begin{equation}
    \label{eqn:chap2-optimal-bellman-operator}
    \contractop^*_V V(s) = \max_{a \in A}\Big[r(s, a) + \gamma \mathbb{E}_{s' \sim \mathcal{T}(s' | s, a)}\brackb{V(s')} \Big]
\end{equation}
Note that it is related to optimal Bellman equation represented in equation \ref{eqn:chap2-optim-Q-and-V}. In this case, we update the the value function toward the action that yields highest value given only one update. This is, indeed, a contraction mapping, thus repeatedly updating the value function by optimal value Bellman operator will give us optimal value function. The proof will be similar to the proof done in section above. Finally, given the optimal value function, one can calculate the optimal action value function, and optimal policy. 

\subsection{Q Iteration}
\label{sec:chap2-Q-iter}
Now, instead of using value function as in value iteration, it is always desirable for us to estimate optimal action value function $Q^*(s, a)$ direction and without having to calculate transition function expectation. By using the same technique, one can define the optimal action value Bellman operator as 
\begin{equation}
    \contractop^*_Q Q(s, a) = r(s, a) + \gamma \mathbb{E}_{s' \sim \mathcal{T}(s' | s, a)}\brackb{\max_{a\in A} Q(s', a)} 
\end{equation}
this operator is indeed a contraction mapping. The proof is similar compare to other algorithms. However, there is more to this. Since there is no need to explicitly consider the transition function, we defined the following update rule given iteartion $k$ and $(s_t, a_t)$ experience:
\begin{equation}
\label{eqn:chap2-Q-stochastic-update}
    Q^{(k+1)}(s_t, a_t) \leftarrow (1-\alpha_t) Q^{(k)}(s_t, a_t) + \alpha\brackb{r(s_t, a_t) + \gamma \max_{a'} Q^{(k)}(s_{t+1}, a')}
\end{equation}
where $0 < \alpha < 1$. Following the theorem 1 of \cite{jaakkola1994convergence}, which is also a basis of proofing Q-Iteration in later section states that:
\begin{theorem}
\label{thm:updaate-stochastic}{\cite{jaakkola1994convergence}}
    The process $\Delta_{t+1} = (1-\alpha_t(x)) \Delta_t + \alpha_t(x) F_n(x) $ will converge to zero if and only if The state space is finite, $\sum_t \alpha_t(x) = \infty$, $\sum_t \alpha_t(x)^2 < \infty$, $\left\|\mathbb{E}\brackb{F_n(x) \big| P_n} \right\| \le \beta \|\Delta_n\|$ where $\beta \in (0, 1)$ and  $\operatorname{Var}\brackb{F_n(x) \big| P_n} \le C\bracka{1 + \|\Delta_n\|}^2$, where $P_n$ is a collection of histories. 
\end{theorem}
By subtracting optimal Q function at both ends i.e $\Delta_t = Q^{(k)}(s_t, a_t) - Q^*(s_t, a_t)$ and $F_n(x) = r(s_t, a_t) + \gamma \max_{a'} Q^{(k)}(s_{t+1}, a') - Q^*(s_t, a_t)$, we can proof that the update in equation \label{eqn:chap2-Q-stochastic-update} will converge to the optimal Q-value. Given Theorem \ref{thm:updaate-stochastic}, we can now change any contraction mapping into a stochastic update rule and still maintains the convergence property. We will use this throughout the text.

\section{Other Related Algorithms}
\label{sec:chap2-other-algos}
This is the list of the algorithms that will be introduced in the following chapters: 
\begin{itemize}
    \item \hyperref[chapter:chap3]{Chapter 3}: Regularized Opponent Model with Maximum Entropy Objective (ROMMEO) \cite{tian2019regularized}, Probabilistic Recursive Reasoning (PR2) \cite{wen2019probabilistic}, and Balancing Q-learning \cite{grau2018balancing}.
    \item \hyperref[chapter:chap5]{Chapter 5}\footnote{\hyperref[chapter:chap4]{Chapter 4} doesn't require any additional introduce but we assuming the readers have read \hyperref[chapter:chap3]{Chapter 3}}:  Maximum a Posteriori Policy Optimization (MPO) \cite{abdolmaleki2018maximum}, VIREL \cite{fellows2019virel}, and its on-policy variance V-MPO \cite{song2019v}. 
    \item \hyperref[chapter:chap6]{Chapter 6}: Soft Hierarchical reinforcement learning \cite{igl2019multitask, lobo2019soft} and Delayed communication or Dynamic policy termination Q-learning \cite{han2019multi}.
    \item \hyperref[chapter:chap7]{Chapter 7}: Inverse reinforcement learning starting with Generative Adversarial Imitation Learning (GAIL) \cite{ho2016generative}, Adverserial Inverse Reinforcement Learning (AIRL) \cite{fu2017learning} with its connection between Generative Adversarial Network (GAN) \cite{goodfellow2014generative, finn2016connection}. Furthermore, we have unified view of adversarial imitation learning via $f$-GAN \cite{nowozin2016f} formulation \cite{ghasemipour2019divergence, ke2019imitation}. Also, we will also consider context awareness reinforcement algorithm \cite{yu2019meta, rakelly2019efficient}, and finally, multi-agent imitation learning including Multi-Agent GAIL (MAGAIL) \cite{song2018multi}, Multi-agent Adversarial Inverse Reinforcement Learning (MA-AIRL) \cite{yu2019multi} and, Decentralized Adversarial Imitation Learning algorithm with Correlated policies \cite{liu2020multi}.Last but not least the probabilistic interpretation of GAN, which includes Bayesian Generative Adversarial Imitation Learning \cite{jeon2018bayesian}, Importance weight GAN (IWGAN) \cite{hu2017unifying}, and $\alpha$-GAN \cite{rosca2017variational}, which is an extension of Adversarial autoencoders (AAE) \cite{makhzani2015adversarial}\footnote{As mentioned before, this chapter will be part-survey and part proposal to the new multi-agent imitation learning}.
    \item \hyperref[chapter:chap8]{Chapter 8}: Generalized Recursive Reasoning (GR2) \cite{wen2019multi}
    \item \hyperref[chapter:chap9]{Chapter 9}: Variational Sequential Monte Carlo, Deep Variational Reinforcement Learning \cite{igl2018deep, shvechikovjoint} and Sequential Variational Soft Q-Learning (SVQN) \cite{huangsvqn}. They are the basis for representing belief, which is a basis for public belief MDP \cite{nayyar2013decentralized} and Bayesian Action Decoder (BAD) \cite{foerster2018bayesian}. 
\end{itemize}

% \section{Extensions to control as inference frameworks}
% After we have introduced control and inference framework, we will now present current progress of reinforcement learning made based on this framework. We will start with hierarchical agent, which is simplest extension to the framework. After this, we will look on how can be represents the unknown environment with belief over the state, which would requires us to formulate the problem into partially observable Markov decision process (POMDP). Finally, we will take a slight turn and examine current work on information theoretic inspired single agent reinforcement learning algorithms.

% \section{Multi-agent reinforcement learning}
% In this section, we will start formulating Multi-agent reinforcement learning problems, by first introducing a formulation of decentralized partially observable Markov decision process (DEC-POMDP), introducing some game theoretic concepts, with more traditional algorithms including Nash-Q \cite{hu2003nash}, MADDPG \cite{lowe2017multi}, COMA \cite{foerster2018counterfactual}, where some of them will be our baseline for our algorithms.

% \section{Multi-agent with probabilistic inference framework}
% Finally, we will present current algorithms developed by re-interpretation multi-agent reinforcement learning problem as control-as-inference framework. We will present 4 algorithms based on this: PR2 \cite{wen2019probabilistic}, ROMMEO \cite{tian2019regularized}, GR2 \cite{wen2019multi}, and Balancing-Q \cite{grau2018balancing}. All of them will be examined in full details because they lays a foundations to main result of this thesis. We will end this chapter by shinning light into incompatibility between these algorithms, which will solve in the later chapters. 