\label{chapter:chap4}
\epigraph{\correctquote{The political is the most intense and extreme antagonism, and every concrete antagonism becomes that much more political the closer it approaches the most extreme point, that of the friend-enemy grouping.}}{The Concept of the Political (1932) \\ by Carl Schmitt}

\begin{miniabstract}
After we have a unified view on MAPI framework, which we have reinterpret Balancing Q-learning \cite{grau2018balancing} as a probabilistic inference. We are now going to consider a more general setting, when the opponent's reward isn't proportional to agent's reward. However, we will show that one can't stray away from the use of $\beta$ variable during the training. We have split this section into a chapter, since, the main goal of the previous chapter is to \correctquote{understand} Balancing Q-learning, while this chapter we will try to extends the algorithm to be as general as possible. The structure of this chapter and any chapter afterward going to be the derivation of algorithm and its neural network implementation.
\end{miniabstract}

\section{General Sum Extension to Balance Q-learning}
\label{sec:chap4-gen-extension}
Let's consider the $Q(s_t, a^{i}_t)$ from our probabilistic Balancing Q-learning discussed in section \ref{sec:chap3-prob-balancing-q}. As we have mentioned, the value of the agent is based on the fraction between agent's $\beta^i$ and opponent's $\beta^{-i}$ as we shown in the original Balancing Q-learning that this is enough for us to convert opponent's value function to the agent's value function. However, this wouldn't be the case for a general sum game. Nonetheless, the $\beta$ still being very important since we can use its sign to convey whether our opponent is a friend or foe. We shall use it to our advantage. Note that it still not the case that we can \correctquote{switch side}, since we will treat $\beta$s to be hyper-parameters. Now, let's consider the opponent model's policy, we shall assume similar case to section \ref{sec:chap3-prob-balancing-q}, but instead, we are going to consider the case where the opponent's reward isn't the same as the agent i.e 
\begin{equation}
\begin{aligned}
\label{eqn:chap3-prob-opponent}
    &\rho_{\phi}(a^{-i}_t | s_t, a^i_t) = \frac{\exp(Q^{-i}(s_t, a^i_t, a^{-i}_t))P_\prior(a^{-i}_t | s_t, a^i_t)}{\exp(Q^{-i}(s_t, a^{i}_t))} \\
    \text{ where }&Q^{-i}(s_t, a^{i}_t) = \log \int \exp(Q^{-i}(s_t, a^i_t, a^{-i}_t))P_\prior(a^{-i}_t | s_t, a^i_t) \dby a^{-i}_t \\
    &Q^{-i}(s_t, a^i_t, a^{-i}_t) = r^{-i}(s_t, a^i_t, a^{-i}_t) + \gamma \mathbb{E}_{s_{t+1}, a^{i}_{t+1}\sim\pi_\theta}\brackb{Q^{-i}(s_{t+1}, a^{i}_{t+1})}
\end{aligned}
\end{equation}
Now for the agent, we should consider the value of $Q^{i}(s_t, a^{i}_t)$, which should be the soft-max given the opponent model's action of $Q^{i}(s_t, a^{i}_t, a^{-i}_t)$

\section{Implementation of The General Sum Extension}
\label{sec:chap4-implementation}