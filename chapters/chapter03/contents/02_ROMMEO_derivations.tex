Now, we shall consider the derivation of ROMMEO using the technique introduced in \cite{levine2018reinforcement}, which is shown briefly in section \ref{sec:chap2-derivation-soft-closed-form}. The working is partially covered in \cite{yu2019multi, wen2019multi} however a full derivation is required for the understanding of the algorithm. We will base our derivation for ROMMEO \cite{tian2019regularized}, while the same technique can be applied to PR2 \cite{wen2019probabilistic}. ROMMEO assumes that the joint probability between 2 agents, which can be factorized as
\begin{equation}
    \pi(a, a^{-i} | s) = \pi(a | a^{-i}, s) \rho(a^{-i} | s)
\end{equation}
factorizing in other ways yields PR2. We will start by drawing the graphical model of joint probabilities that we want to approximate, and is depicted in Figure \ref{fig:chap3-approximated-ROMMEO}. 
\begin{figure}[ht]
    \begin{minipage}[t]{0.5\linewidth}
    \centering
    \begin{tikzpicture}[latent/.append style={minimum size=1.5cm}]
        \node[obs] (ot) {$\mathcal{O}^i_t, \mathcal{O}^{-i}_t$};
        \node[latent, below=of ot] (at) {$a^{-i}_t$};
        \node[latent, below=of at] (st) {$s_t$};
        \node[latent, below=of st] (a1t) {$a^i_t$};
        % \node[obs, below=of a1t] (o1t) {$\mathcal{O}^i_t$};
        
        
        % \path[->]  (o1)  edge   [bend left=45] node {} (a);
        % \edge{o1t} {a1t}
        \edge {st} {at, a1t}
        \edge {at} {ot}
        % \path[->]  (o1t)  edge   [bend left=60] node {} (ot);
        \path[->]  (a1t)  edge   [bend left=45] node {} (ot);
        \path[->]  (a1t)  edge   [bend left=30] node {} (at);
        \path[->]  (st)  edge   [bend left=30] node {} (ot);
        
        \node[obs, right=3cm of ot] (ot1) {$\mathcal{O}^i_{t+1}, \mathcal{O}^{-i}_{t+1}$};
        \node[latent, below=of ot1] (at1) {$a^{-i}_{t+1}$};
        \node[latent, below=of at1] (st1) {$s_{t+1}$};
        \node[latent, below=of st1] (a1t1) {$a^i_{t+1}$};
        % \node[obs, below=of a1t1] (o1t1) {$\mathcal{O}^i_{t+1}$};
        
        % \edge{o1t1} {a1t1}
        \edge {st1} {at1, a1t1}
        \edge {at1} {ot1}
        % \path[->]  (o1t1)  edge   [bend right=60] node {} (ot1);
        \path[->]  (a1t1)  edge   [bend right=45] node {} (ot1);
        \path[->]  (a1t1)  edge   [bend right=30] node {} (at1);
        \path[->]  (st1)  edge   [bend right=30] node {} (ot1);
        
        \path[->]  (st)  edge   [] node {} (st1);
        \path[->]  (at)  edge   [] node {} (st1);
        \path[->]  (a1t)  edge   [] node {} (st1);
        
        \node[latent, right=5cm of st1] (stn) {$s_{t+n}$};
        \path[->]  (st1)  edge [] node {} (stn);
    \end{tikzpicture}
    \end{minipage}%
    \begin{minipage}[t]{0.5\linewidth}
    \caption{Graphical model that we want to approximate. This is based on the joint probability provided in ROMMEO paper, with slight modification. The authors assume that the opponent we are playing against is optimal}
    \label{fig:chap3-approximated-ROMMEO}
    \end{minipage}
\end{figure}
Before we move on, please note that this is all within an agent calculation, while the opponent's policy will have the same process. Given this, we can show that the prior joint probability is equal to the following:
\begin{equation}
    \begin{aligned}
        P(s_{1:T}, &a^i_{1:T}, a^{-i}_{1:T}, \optim^{-i}_{1:T} = 1, \optim^{i}_{1:T} = 1) \\
        &= P(s_0) \prod^T_{t=0} P_{\prior}(a^i_t | s_t, a^{-i}_t) P_{\prior}(a^{-i}_t | s_t) P(s_{t+1} | s_t, a^i_t, a^{-i}_t) P(\optim^{-i}_t = 1, \optim^{i}_t = 1 | s_t, a^i_t, a^{-i}_t)
    \end{aligned}
\end{equation}
where the optimality variable is defined as 
\begin{equation}
\begin{aligned}
    &P(\optim_t^{-i} = 1, \optim^{i}_t = 1 | s_t, a^i_t, a^{-i}_t) \propto \exp \bracka{ \beta r(s_t, a^i_t, a^{-i}_t)  } \\
    &P(\optim_t^{-i} = 1, \optim^{i}_t = 1 | s_t, a^i_t, a^{-i}_t) = P(\optim_t^{-i} = 1 |\optim^{i}_t = 1, s_t, a^i_t, a^{-i}_t) P(\optim^{i}_t = 1 | \optim^{-i}_t = 1, s_t, a^i_t, a^{-i}_t)
\end{aligned}
\end{equation}
This definition is different from the one proposed in \cite{tian2019regularized}. However, we would like to point out that the opponent model has been optimized \textit{after} the agent's policy, which will show in a later section. Therefore, during the optimization of the opponent model, it has assumed that the agent's policy is optimal, hence the factorization of the optimality. Furthermore, the factorization of the optimality random variables above is a mere representation because the distinction between agent's and its opponent model lies during the calculation of the solution and not during the inference. Now for the variational joint probabilities that we want to optimize on the graphical model is depicted in Figure \ref{fig:chap3-variational-ROMMEO}.
\begin{figure}[ht]
    \begin{minipage}[t]{0.5\linewidth}
    \centering
    \begin{tikzpicture}[latent/.append style={minimum size=1.0cm}]
        \node[latent] (at) {$a^i_t$};
        \node[latent, below=of at] (st) {$s_t$};
        \node[latent, below=of st] (a1t) {$a^{-i}_t$};
        
        % \edge{o1t} {a1t}
        \edge {st} {at, a1t}
        % \edge {at} {ot}
        % \path[->]  (o1t)  edge   [bend left=60] node {} (ot);
        % \path[->]  (a1t)  edge   [bend left=45] node {} (ot);
        \path[->]  (a1t)  edge   [bend left=30] node {} (at);
        % \path[->]  (st)  edge   [bend left=30] node {} (ot);
        
        % \node[obs, right=3cm of ot] (ot1) {$\mathcal{O}^i_{t+1}$};
        \node[latent, right=3cm of at] (at1) {$a^i_{t+1}$};
        \node[latent, below=of at1] (st1) {$s_{t+1}$};
        \node[latent, below=of st1] (a1t1) {$a^{-i}_{t+1}$};
        % \node[obs, below=of a1t1] (o1t1) {$\mathcal{O}^{-i}_{t+1}$};
        
        % \edge{o1t1} {a1t1}
        \edge {st1} {at1, a1t1}
        % \edge {at1} {ot1}
        % \path[->]  (o1t1)  edge   [bend right=60] node {} (ot1);
        % \path[->]  (a1t1)  edge   [bend right=45] node {} (ot1);
        \path[->]  (a1t1)  edge   [bend right=30] node {} (at1);
        % \path[->]  (st1)  edge   [bend right=30] node {} (ot1);
        
        \path[->]  (st)  edge   [] node {} (st1);
        \path[->]  (at)  edge   [] node {} (st1);
        \path[->]  (a1t)  edge   [] node {} (st1);
        
        \node[latent, right=5cm of st1] (stn) {$s_{t+n}$};
        \path[->]  (st1)  edge [] node {} (stn);
    \end{tikzpicture}
    \end{minipage}%
    \begin{minipage}[t]{0.5\linewidth}
    \caption{Graphical model that we are going to optimize our policies on. This has almost the same structure as the version that we want to approximate. However, we don't care about the optimality of the agent itself, as we want to approximate its policy (denote as $\pi$) and the its opponent model (denote as $\rho$)}
    \label{fig:chap3-variational-ROMMEO}
    \end{minipage}
\end{figure}
The joint probability of variational distribution can be calculated as following 
\begin{equation}
    q(s_{1:T}, a_{1:T}^i, a^{-i}_{1:T}) = P(s_0) \prod^T_{t=0} \pi_{\theta}(a^i_t | s_t, a^{-i}_t) \rho_{\phi}(a^{-i}_t | s_t) P(s_{t+1} | s_t, a^i_t, a^{-i}_t)P(\optim^{-i}_t = 1)
\end{equation}
We would like to solve the following optimization problem 
\begin{equation}
    \argmin{\pi, \phi \in \Pi \times \Phi}  \kl\bracka{q(s_{1:T}, a_{1:T}^i, a^{-i}_{1:T}) \Big\| P(s_{1:T}, a^i_{1:T}, a^{-i}_{1:T}|\optim^i_{1:T} = 1  \optim^{-i}_{1:T} = 1) }
\end{equation}
This leads to optimizing the following ELBO. The derivation is shown in the appendix \ref{appx:chap3-ROMMEO-ELBO}, which translates the problem into a maximization problem of the following objective
\begin{equation}
\label{eqn:chap3-ROMMEO-ELBO}
    \argmax{\pi, \phi \in \Pi \times \Phi} \mathbb{E}_{q}\brackb{ \sum^T_{t=0} \gamma^t\bracka{\beta r(s_t, a^i_t, a^{-i}_t)  - \log \frac{\pi_{\theta}(a^i_t | s_t, a^{-i}_t)}{P_{\prior}(a^i_t | s_t, a^{-i}_t) } - \log \frac{\rho_{\phi}(a^{-i}_t | s_t)}{P_{\prior}(a^{-i}_t | s_t)}}}
\end{equation}
We will start off by considering the last time step, as we will progress backwards in time. Considering the last time step:
\begin{equation}
\label{eqn:training-obj-last-time}
    \mathbb{E}_{q} \brackb{ \beta r(s_T, a^i_T, a^{-i}_T)  - \log \frac{\pi_{\theta}(a^i_T | s_T, a^{-i}_T)}{P_{\prior}(a^i_T | s_T, a^{-i}_T) } - \log \frac{\rho_{\phi}(a^{-i}_T | s_T)}{P_{\prior}(a^{-i}_T | s_T)}}
\end{equation}
We can show that the optimal policy is equal to 
\begin{equation}
\begin{aligned}
    &\pi_{\theta}(a^i_T | s_T, a^{-i}_T) = \frac{\exp \bracka{\beta r(s_t, a^i_T, a^{-i}_T)}P_{\prior}(a^i_T | s_T, a^{-i}_T)}{\exp\bracka{Q(s_T, a^{-i}_T)}} \\
    \text{where }& Q(s_T, a^{-i}_T) = \log \int \exp \bracka{\beta R(s_T, a^i_T, a^{-i}_T}P_{\prior}(a^i_T | s_T, a^{-i}_T)  \dby a^i_T
\end{aligned}
\end{equation}
The proof will be presented in the appendix \ref{appx:chap3-ROMMEO-agent-last-time}. This part is where we can assume that the agent policy is optimal and allow us to calculate the opponent model objective. By plugging the optimal policy back, we are left with the opponent's objective:
\begin{equation}
    \mathbb{E}_{ \rho_{\phi}(a^{-i}_T | s_T) P(s_T)}\brackb{ Q(s_T, a^{-i}_T) - \log \frac{\rho_{\phi}(a^{-i}_T | s_T)}{P_{\prior}(a^{-i}_T | s_T)} }
\end{equation}
which will give us the optimal opponent model at time step $T$ to be as the following:
\begin{equation}
\begin{aligned}
    &\rho_{\phi}(a^{-i}_T | s_T)  = \frac{ \exp\bracka{Q(s_T, a^{-i}_T)}P_{\prior}(a^{-i}_T | s_T) }{\exp\bracka{V(s_T)}}  \\
    \text{where }& V(s_T) = \log \int \exp\bracka{Q(s_T, a^{-i}_T)}P_{\prior}(a^{-i}_T | s_T) \dby a^{-i}_T
\end{aligned}
\end{equation}
All the proofs including the opponent's objective is in appendix \ref{appx:chap3-ROMMEO-opponent-last-time}. Given both agent and its opponent's model policy, we can consider the message passed to the time steps before, and we can see that we are left with
\begin{equation}
    \mathbb{E}_{P(s_T)}\brackb{V(s_T)}
\end{equation}
This quantity will be passed down to the later time. Now for time $t$, the objective that we are optimizing becomes 
\begin{equation}
    \mathbb{E}_{q} \Bigg[ \underbrace{\beta r(s_t, a^i_T, a^{-i}_T) + \gamma \mathbb{E}_{P(s_{t+1} | s_t, a^i_t, a^{-i}_t)}  \brackb{V(s_{t+1})}}_{Q(s_t, a^i_t, a^{-i}_t)}  - \log \frac{\pi_{\theta}(a^i_T | s_t, a^{-i}_t)}{P_{\prior}(a^i_t | s_t, a^{-i}_t) } - \log \frac{\rho_{\phi}(a^{-i}_t | s_t)}{P_{\prior}(a^{-i}_t | s_t)} \Bigg]
\end{equation}
We can see that this leads to very similar problem as the one we have solved before. Therefore, by using the same proving process, we can derive the optimal agent's policy and optimal opponent model policy to be 
\begin{equation}
\begin{aligned}
    \label{eqn:chap3-final-ROMMEO-policy}
    \pi_{\theta}(a^i_t | s_t, a^{-i}_t) = \frac{\exp \bracka{Q(s_t, a^i_t, a^{-i}_t)}P_{\prior}(a^i_t | s_t, a^{-i}_t) }{\exp\bracka{Q(s_t, a^{-i}_t)}} \quad \quad \rho_{\phi}(a^{-i}_t | s_t) = \frac{ \exp\bracka{Q(s_t, a^{-i}_t)}P_{\prior}(a^{-i}_t | s_t) }{\exp\bracka{V(s_t)}}
\end{aligned}
\end{equation}
where the analogous "Bellman equation" is the following (with the value and action value functions being)
\begin{equation}
    \label{eqn:chap3-final-ROMMEO}
    \begin{aligned}
        &Q(s_t, a^i_t, a^{-i}_t) = \beta r(s_t, a^i_t, a^{-i}_t) + \gamma\mathbb{E}_{P(s_{t+1} | s_t, a^i_t, a^{-i}_t)} \brackb{V(s_{t+1})} \\
        \text{where }&Q(s_t, a^{-i}_t) = \log \int \exp \bracka{Q(s_t, a^i_t, a^{-i}_t}P_{\prior}(a^i_t | s_t, a^{-i}_t)  \dby a^i_t \\ 
        &V(s_t) = \log \int \exp\bracka{Q(s_t, a^{-i}_t)}P_{\prior}(a^{-i}_t | s_t) \dby a^{-i}_t \\
    \end{aligned}
\end{equation}    
This concludes the derivation for ROMMEO in its most general form. If we consider PR2, then we have to maximize opponent model first, by using a similar process.

