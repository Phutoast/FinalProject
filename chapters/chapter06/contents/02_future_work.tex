\label{sec:chap6-future-work}

We have tackled two extensions to our MAPI framework, which allows the agent to develop a fundamental ability for any multi-agent system, which are the ability to communicate and the ability to reason under uncertainty. As shown in chapter \ref{chapter:chap4} and chapter \ref{chapter:chap5}, the algorithm presented is slightly too large to implement via standard machine. One of the immediate task to complete in a future would be to reduce the size of the algorithm so that it is manageable. However, this isn't the primary goal of this thesis, in which our goal is to show that the multi-agent problem can be reduced down to the probabilistic model, which can be solved with ease. The logical next step would be to apply this framework on other problems that are untackled in this thesis, including a unified view of adversarial imitation learning and context-aware agent in a multi-agent setting, and more generalized recursive reasoning that includes competitive games with unknown environment state. Furthermore, multi-tasking and distilling of multi-agent policy can also be exciting problems as they are related to control as probabilistic inference framework or even lifelong learning algorithm for multi-agent problems.

Looking back, investigating the effects of entropy regularization in the opponent model and agent through the lens of exploration can be fruitful, as shown in \cite{mahajan2019maven} that some of the multi-agent algorithms do suffer from exploration problem. Would this also occur in entropy regularized MARL algorithms? How the regularization affects the training trajectory of the agent? These are importance questions that require further investigation that are not investigated in this work.

In conclusion, there are two primary directions that one can take. Firstly, one can expand the family of MAPI algorithms by reinterpreting the single-agent algorithm and problems as a probabilistic model. Or, secondly, one can look inward on analyzing the performance that entropy regularization gives to a multi-agent algorithm. 