\label{chapter:intro}
% \epigraph{Let us first look for its quality in states, and then only examine it also in the individual, looking for the likeness of the greater in the form of the less.}{\textit{Plato 369a}}
\begin{miniabstract}
    % In this chapter, we will go through, in more details, the main problems in this thesis, structure of this thesis and the contributions made. At the start, we will provide major drawn backs into the current algorithms in the literature. After this, we will show the current development control-as-inference in single agent domains, which can be extends to multi-agent domain, showing possible profits from resolving the stated problem. Finally, we will expand the whole layout of this thesis.
    In this chapter, we will investigate the problems posted and trying to capture the main theme is this thesis in more details. We will start by giving a brief overview of control as inference framework, and maximum entropy multi-agent reinforcement learning (MEMARL) and provides some of the algorithms in the family. Furthermore, we will show some of the developments of both frameworks (and their problem), and other related algorithms that will be used in our work\footnote{We will simply talk about what these algorithms do but we refers mainly to chapter \ref{chapter:chap2} and other chapters for more thorough treatment.}. Finally, we expand the whole layout of this thesis going, in details, the content of each chapter.  
\end{miniabstract}

\section{Motivation}
\section{Motivation}

\subsection{Incompatibility within MEMARL framework}

In MEMARL family of algorithms, we can divide algorithms into 2 main types: multi-agent learning with probabilistic inference (MAPI) and multi-agent learning with KL-constraint (MAKL). Both types are difference in term of how they arrived at the final algorithm. Notes that KL-divergence with uniform algorithms is the same as entropy (positive or negative depends on the side 2 probability are on). Both of them are algorithms that mainly maximizes its reward with some KL-constrain so that the final policy is some what near the prior, which is directly inspired by control-as-inference single agent framework and bounded rationality. However, there are subtle incompatibility in the way both kind of algorithms are developed. 

We will use ROMMEO \cite{tian2019regularized} as a candidate algorithm that is derived from MAPI perspective, others includes PR2 \cite{wen2019probabilistic} and GR2 \cite{wen2019multi}. MAPI initial goal for development is to facilitate recursive reasoning within agent by representing agent and its opponent to be conditional probability of others action, i.e in ROMMEO case the joint probability is factorized as
\begin{equation*}
    \pi(a^{i}, a^{-i} | s) = \pi(a^{i} | s, a^{-i}) \rho(a^{-i} | s)
\end{equation*}
where $a^i$ is the agent's action and $a^{-i}$ is the action of other agents. The main differences for these framework are how factorization is made. The authors introduces an optimality random variable of agent $i$ (which is now defined only for cooperative game) $\mathcal{O}^{i}_t$ that indicates whether its acts optimal or not at time $t$, and is defined proportional to exponential of joint reward.
\begin{equation*}
    P(\mathcal{O}^{i}_t = 1 | \mathcal{O}^{-i}_t = 1) \propto \exp\left( \beta r(s_t, a^i_t, a^{-i}_t) \right)
\end{equation*}
where $\beta$ is called temperature variable. Note that defining an optimality random variable is a common technique in a super-set of control-as-inference framework, as we want to maximizing the probability of this throughout the whole time step. This can be achieved using variational inference. This leads agent's objective to be maximizing the reward with the constraint controlled by $\beta$, so that the final policy is near to the prior. The opponent model's objective is to maximize the reward with similar KL-constrain as agent's policy. The beauty lies in the fact that the agent can also optimizes its opponent model so that it is "optimal" rather than just represent other player's policy which partially solve the problem of centralized training. However, this also posts a \textit{huge} problem, since in zero-sum case the agent will optimizes its model so that its acts as if the opponent is maximizing its own reward, which is completely opposite behavior to what we desire.

For MAKL algorithm, the most prominent algorithm is Balancing-Q learning \cite{grau2018balancing}, in which the authors draws an inspiration from KL-constrained single agent learning, which leads to following loss 
\begin{equation*}
    \mathbb{E}\left[ \sum^T_{t=0} \gamma^t \left(R(s_t, a_t, a^{-i}_t) - \frac{1}{\beta^i} \log\frac{\pi(a_t|s_t)}{P_{\operatorname{prior}}(a_t|s_t)} - \frac{1}{\beta^{-i}}\log \frac{\rho(a^{-i}_t|s_t)}{P_{\operatorname{prior}}(a^{-i}_t|s_t)} \right) \right]
\end{equation*}
This is seen as putting difference constrains $\beta^i, \beta^{-i}$ on both agents and opponents, respectively, in which. The the case where $\beta^{-i}$ is negative and $\beta^i$ is positive the resulting algorithms are acting in zero-sum manner. And, if both of them are positive, then the game becomes cooperative. This is very interesting, since this essentially filled the gap within in MAPI framework. However, there isn't any obvious way to further extend this algorithms or the way to map this approach to MAPI framework.

In summary, MAPI provided us with principled approach to develop multi-agent reinforcement algorithms, however, it lacks expressiveness to solve zero-sum game. On the other hand, MAKL provided us with baked formulation that allows us to works on zero-sum game, which can be a difficulty when we want to extends the current MEMARL framework. We will need to find a way so to develop an algorithm that is similar to one derived from MAKL, but is based on MAPI approach. For a more details on all algorithms including interpretation of temperature variable $\beta$ and implementations, please refer to \hyperref[chapter:chap1]{background chapter}.

\Phu{Adding Inverse-RL problem + have to restucture abit so that it shows importance of this}

\subsection{Plausible extensions}
Control-as-inference framework has been very fruitful on interpreting some reinforcement learning algorithms techniques, such as hierarchical control, inferring hidden state within partial observable markov decision process (POMDP), and information theoretic extensions, mainly empowerment for exploration etc. as probabilistic inference problem. In this section, we will briefly describe on how both additional features can be interpreted and be merged into multi-agent problem.

Start of with hierarchical control, which we will based on algorithm MSOL \cite{igl2019multitask}, presents an base agent $\pi(a_t | s_t, z_t)$ coupled with higher-level agent $\pi^H(z_t |s_t, z_{t-1}, b_t)$ and terminal agent $\pi^T(b_t | s_t, z_{t-1})$. The base-agent will receives an instruction from higher-level agent to acts on certain behavior defined by $z_t$, and the higher-level agent can switch base-agent behavior, when the terminal agent sends a message to change $b_t$ (or to terminal the current behavior). Given this formulation, the agent is able reason in extended temporal setting, while being able to perform multi-task learning, which speeds up learning. This will be very important extension to multi-agent reinforcement learning. Furthermore, by having a higher-agent manipulating base agent by generating $z_t$ is similar to MAVEN \cite{mahajan2019maven}, a new state-of-the-art multi-agent reinforcement learning algorithm, which is shown empirically to helps agent explore by letting it perform difference modes of sequence of action. 

For the more fundamental question of how to represents belief, we turn to a problem of POMDP. Normally, one would use recurrent neural network, such as LSTM \cite{hochreiter1997long} or GRU \cite{cho2014properties} to keep track of history of observations, and compute accumulated information to act. However, this approach lacks representation of epistemic uncertainty in the current state of the world. One promising approach is to quantify uncertainty via particle filtering \cite{igl2018deep} where each particle is a 3-element tuple of $(z_t, h_t, w_t)$, where $z_t$ is latent state representation (since we don't know the actual state), $h_t$ is the hidden representation of RNN, and $w_t$ is the weight of each particle, which acts as uncertainty quatifier. Given the particle, the agent can use them to arrived at the final action $a_t$. We can interpret the weight as uncertainty quantifier. The current work \cite{shvechikovjoint}  has interpret this approach as a probabilistic inference with sequential monte carlo sampling. Given the tools developed so far, there are 2 main extensions to multi-agent algorithms that we can do: POMDP-MARL, quantify uncertainty of other agents behavior (via "ensemble" of opponent models) and inferring belief of other agents (theory of mind). For all of the extensions, the agents are required to have the understanding of the observation other agents receive. This problem can be framed as multi-viewed problem \cite{li2019multi}, which we can employ developed generative models, such as generative query network \cite{eslami2018neural}. 

Finally, we employ the current development of information-theoretic based single-agent reinforcement learning problem, which has been shown to be a more generalization of most of control-as-inference framework \cite{grau2018soft, leibfried2019unified}. This framework is based on the ability to optimizes prior term, leading to following optimization problem (in a brief form)
\begin{equation*}
    \max_{\pi, \rho} \mathbb{E} \Bigg[ \sum^\infty_{i=0} \gamma^{t} \bigg( r(s_t, a_t) - \frac{1}{\beta}\frac{\pi(a_t|s_t)}{\rho(a_t)} \bigg) \Bigg]
\end{equation*}
The author has shown that by optimizing prior this way, one recover reward maximization algorithm with mutual-information (MI) constraint, and the author has shown that this framework can promote exploration. In term of extension to multi-agent reinforcement learning algorithm, we can have better prior over the opponent model with better exploration. 

\section{Outline}
\label{sec:chap1-Outline}
This outline in this thesis are as follow list. Chapter \ref{chapter:chap4} onward are the extensions we make given our understanding of MEMARL framework explored in chapter \ref{chapter:chap3}.
\begin{itemize}
    \item \textbf{\hyperref[chapter:chap2]{Chapter 2} (Background)}: We will go through the tools necessary to develop the algorithms and formulation of the problem in multi-agent reinforcement learning (MARL)\footnote{We will include introduction to chapter specific algorithms/interpetatation (for example the explanation of Baysian Action Decoder \cite{foerster2018bayesian} and DVRL as probabilistic graphical model \cite{shvechikovjoint} are in chapter \ref{chapter:chap9} ) to that chapter. This would reduce the space of the background and make the information easy to search. The intention of this chapter is to the readers understand the current landscape of MARL algorithms.}, which includes 
    \begin{itemize}
        \item Variational inference techniques, Expectation Maximization (EM), and Stein Variational Gradient Descent (SVGD) \cite{liu2016stein}, which will be used in developing and implementing most of the algorithms.
        \item Single agent reinforcement learning: MDP and POMDP formulation, value iteration, and its contraction proof + improvement theorem\footnote{We would like to restate those proof because we will encounter similar proof throughout the thesis}
        \begin{itemize}
            \item Basic Deep reinforcement learning, including
            \begin{itemize}
                \item  Deep Q-learning \cite{mnih2015human} (and their variances such as double Q-learning \cite{van2016deep}).
                \item Policy Gradient \cite{sutton2000policy} (and their variance such as advantage actor critic (A2C) \cite{mnih2016asynchronous}\footnote{The citation mainly develop asynchronous update hence the name Asynchronous A2C or A3C. The synchronous version is presented in \url{https://openai.com/blog/baselines-acktr-a2c/}}, Trust Region Policy Optimization (TRPO) \cite{schulman2015trust} and Proximal Policy Optimization (PPO) \cite{schulman2017proximal}).
                \item Deep Deterministic Policy Gradient (DDPG) \cite{lillicrap2015continuous} and Twin Delayed DDPG (TD3) \cite{fujimoto2018addressing}.
            \end{itemize}
            \item Control as inferences frameworks, soft Q-learning \cite{haarnoja2017reinforcement}, and soft Actor-critic \cite{haarnoja2018softa, haarnoja2018softb}
            \item We will, also, includes recent works on bounded rationality as an alternative explanation of reinforcement learning with KL-divergence constraint \cite{ortega2013thermodynamics, tim2015bounded}.
        \end{itemize}
        % \item Extensions to control as inference frameworks: Hierarchical reinforcement learning \cite{igl2019multitask}, representing belief in POMDP \cite{igl2018deep}, and inverse-reinforcement learning \cite{fu2017learning}. 
        \item Multi-agent reinforcement learning (MARL) algorithms: 
        \begin{itemize}
            % \item We will go through DEC-POMDP \cite{bernstein2002complexity} formulation of MARL problem and other related MDPs for multi-agent. This would include the notion of stochastic game and Nash Equilibrium.
            \item We will go through stochastic game \cite{shapley1953stochastic}. This would include the notion of Best Response and Nash Equilibrium \cite{nash1950equilibrium}.
            \item We also going to give overviews of value learning based multi-agent reinforcement learning algorithms, which is related to some of the algorithms in the latter chapters. This includes: Nash Q-learning \cite{hu2003nash} and Friend-or-Foe Q-learning \cite{littman2001friend}.
            \item we will survey importance deep multi-agent reinforcement learning algorithms, including: Multi-Agent DDPG \cite{lowe2017multi}, Counterfactual Multi-Agent Policy Gradients (COMA-PG) \cite{foerster2018counterfactual}, Value Decomposition Network \cite{sunehag2017value}, QMIX \cite{rashid2018qmix}, and Multi-Agent Variational Exploration (MAVEN) \cite{mahajan2019maven}. 
        \end{itemize}
    \end{itemize}
    \item \textbf{\hyperref[chapter:chap3]{Chapter 3} (Unified view)}: In this chapter, we will consider the full survey of MEMARL framework and try to bring every algorithms into single control as probabilistic inference framework, which is done by examine how each algorithms are derived  and observe the common pattern. This would resulted in probabilistic version of Balancing Q-learning, which acts similar Balancing Q-learning while being based on probabilistic inference framework.
    \item \textbf{\hyperref[chapter:chap4]{Chapter 4} (General Sum Extension)}: Given the probabilistic formulation of Balancing Q-learning learning, we would like to provide an simple extensions to accommodate general sum setting, extending the capacity of Balancing Q-learning.
    \item \textbf{\hyperref[chapter:chap5]{Chapter 5} (EM-Based MARL)}: Now, we shall apply multi-agent problem to EM-Based algorithm for reinforcement learning, which is based, mainly, on VIREL and MPO. This would lead to an interesting algorithm that has been examined in other setting. 
    \item \textbf{\hyperref[chapter:chap6]{Chapter 6} (Hierarchy and Delayed communication MARL)}: By this chapter, we have examine thoroughly the control as probabilistic infernece frameworks' application on multi-agent setting. We will turn toward augmenting agent to be able to reason in temporal abstraction, which leads to delayed option communication framework. As a minor contribution, we also provide soft-option learning algorithm for single agent reinforcement learning based on soft Actor-Critic algorithms.
    \item \textbf{\hyperref[chapter:chap7]{Chapter 7} (Imitation Learning in MARL)}: This chapter will be in slightly difference theme compare to the others. That is because, we are dealing with imitation learning of the opponents and opponent context inference so that our agent cound exploit its knowledge about the opponent. We will so surveying plausible extension to adversarial imitation learning to newly extended framework we proposed. 
    \item \textbf{\hyperref[chapter:chap8]{Chapter 8} (Revisiting Generalized Recursive Reasoning)}: We will turn our attention to recursive reasoning in multi-agent setting. We investigate the current generalized recursive reasoning framework \cite{wen2019multi}, we point out some issues that requires extra attention (notably how the action value function is represented). In turn, we will try to proposed an alternative training of generalized recursive reasoning, which should be more faithful toware control as probabilistic framework. This would require some observation given in chapter \ref{chapter:chap3}. 
    \item \textbf{\hyperref[chapter:chap9]{Chapter 9} (Reinterpretation Public Belief MDP)}: Finally, we will proposed a re-interpretation of public belief MDP via control as inference framework. This would lead to VSMC like algorithm, which allows the agents to reason in counterfactual manner without the need of recursive reasoning.
    \item \textbf{\hyperref[chapter:chap10]{Chapter 10} (Conclusion/Afterword)}: We will reviews our progress so far and conclude with future works and  research direction. 
\end{itemize}
In the appendixes, we will mostly provided a full derivation, for completeness, that are left in main manuscript. We belief that some of the techniques can be useful, thus worth to stated in full form. Appendixes will be divided by the content of each chapters. Finally, we refer to figure \ref{fig:chap0-chap-dependencies} for the dependencies between each chapters.