\documentclass{report}
\usepackage{setspace}
%\usepackage{subfigure}

\pagestyle{plain}
\usepackage{amssymb,graphicx,color}
\usepackage{amsfonts}
\usepackage{latexsym}
\usepackage{a4wide}
\usepackage{amsmath}
\usepackage{cancel}
\usepackage{tcolorbox}
\usepackage{caption}
\usepackage{float}
\usepackage{changepage}
\usepackage{hyperref}


\tcolorboxenvironment{definition}{
  sharp corners,
  boxrule=0.4pt,
  colback=white,
  before skip=\topsep,
  after skip=\topsep,
}

\usepackage{tikz}
\usetikzlibrary{bayesnet}
\usetikzlibrary{arrows}
\usetikzlibrary{calc}
\usetikzlibrary{shadows}
\usetikzlibrary{positioning}

\usepackage{prerex}
\usetikzlibrary{fit}

\usepackage[english]{babel}
\usepackage{blindtext}

\newtheorem{theorem}{Theorem}
\newtheorem{lemma}[theorem]{Lemma}
\newtheorem{corollary}[theorem]{Corollary}
\newtheorem{proposition}[theorem]{Proposition}
\newtheorem{remark}[theorem]{Remark}
\newtheorem{definition}[theorem]{Definition}
\newtheorem{fact}[theorem]{Fact}

\newtheorem{problem}[theorem]{PROBLEM}
\newtheorem{exercise}[theorem]{EXERCISE}
\def \set#1{\{#1\} }

\newenvironment{proof}{
PROOF:
\begin{quotation}}{
$\Box$ \end{quotation}}

\newcommand{\Phu}[1]{{\bf \color{red} [[Phu: #1]]}}
% \newcommand{\Phu}[1]{{}}


\newcommand{\dby}{\ \mathrm{d}}
\newcommand{\kl}{D_{\mathrm{KL}}}
\newcommand{\argmax}[1]{\underset{#1}{\arg\max \ }}
\newcommand{\argmin}[1]{\underset{#1}{\arg\min \ }}
\newcommand{\const}{\text{const.}}
\newcommand{\optim}{\mathcal{O}}
\newcommand{\prior}{\text{prior}}
\newcommand{\bracka}[1]{\left( #1 \right)}
\newcommand{\brackb}[1]{\left[ #1 \right]}
\newcommand{\brackc}[1]{\left\{ #1 \right\}}
\newcommand{\contractop}{\mathcal{T}}
\newcommand*\circled[1]{\tikz[baseline=(char.base)]{
            \node[shape=circle,draw,inner sep=2pt] (char) {#1};}}
\newcommand{\red}[1]{{\color{red} #1}}
\newcommand{\loss}{\mathcal{L}}
\newcommand{\pri}{\operatorname{pri}}
\newcommand{\pub}{\operatorname{pub}}

\newcommand{\grammar}[1]{{\color{blue} #1}}
\newcommand{\factcheck}[1]{{\color{magenta} #1}}

\newcommand{\nats}{\mbox{\( \mathbb N \)}}
\newcommand{\rat}{\mbox{\(\mathbb Q\)}}
\newcommand{\rats}{\mbox{\(\mathbb Q\)}}
\newcommand{\reals}{\mbox{\(\mathbb R\)}}
\newcommand{\ints}{\mbox{\(\mathbb Z\)}}

%%%%%%%%%%%%%%%%%%%%%%%%%%


\title{{{\Huge Multiagent Learning with Bayesian inference}}\\
% {\large Optional Subtitle}\\
}
\date{Submission date: Day May 2020}
\author{Phu Sakulwongtana\thanks{
{\bf Disclaimer:}
This report is submitted as part requirement for the BSc Computer Science at UCL. It is
substantially the result of my own work except where explicitly indicated in the text.
% \emph{Either:} 
The report may be freely copied and distributed provided the source is explicitly acknowledged
% \newline  %% \\ messes it up
% \emph{Or:}\newline
% The report will be distributed to the internal and external examiners, but thereafter may not be copied or distributed except with permission from the author.
}
\\ \\
BSc Computer Science\\ \\
Jun Wang, Philip Treleaven}

\newenvironment{outline} {\renewcommand\abstractname{}\begin{abstract}} {\end{abstract}}

\newenvironment{miniabstract} {\begin{adjustwidth}{1cm}{}} {\end{adjustwidth}}


\begin{document}
 
\onehalfspacing
\maketitle

\begin{abstract}
% In this thesis, we aimed to unify difference probabilistic multi-agent reinforcement learning algorithms into one framework that allows us to extend previous works to operate in more general cases. We first show that the current probabilistic modelling doesn't allow the zero-sum game to be solved, in which we proposed double probabilistic modelling that iterative update the agent and its opponent model with theoretical guarantee such as proof of convergence and opponent modeling improvement theorems. This is analogous to fictitious play, however, instead of best response to the opponent's policy, the proposed algorithm is best responding to opponent's optimal policy. After finalizing the unified framework for all algorithms, we then show that this framework can be extended into hierarchical multi-agent reinforcement learning and can be used to model belief of opponent's private information, thus proving the effectiveness and flexibility of the framework. We test our algorithm on challenging benchmarks against known algorithms with high-dimensional state description, proving empirically the performance of derived algorithms. We hope that framing multi-agent reinforcement learning into probabilistic inference problem will give rise to many novel algorithms, while being flexible to extends into known methods within probabilistic realm.


% In this thesis, we investigate a family of algorithms called maximum entropy multi-agent reinforcement learning (MEMARL), which is derived from variational inferences, and is a direct extension to a well-established control-as-inference framework in single agent reinforcement learning. By framing multi-agent problems as a probabilistic inference, one can includes concepts such as recursive reasoning \cite{wen2019probabilistic, wen2019multi} into a multi-agent reinforcement learning algorithms, while being decentralized training and decentralized execution. However, there are some limitation into such algorithms, notably inability to solve zero-sum game. In order to solve this major drawn back, we will have to step-back and derived an unified views of most algorithms in the literature. Our contributions are mainly as follows: 

% \begin{enumerate}
%     \item Deriving a tools that would let us constructs all the algorithms within MEMARL family of algorithms, which will show the flawed in some of MEMARL algorithms.
%     \item Given these tools, we provided a simple extension to some of MEMARL algorithm, so that they works within zero-sum game, with some theoretical guarantee.
%     \item We will also show that our algorithms works in practice with strong empirical results against various state-of-the-art algorithms.
%     \item After solving the major drawn-back within MEMARL framework, we will borrow techniques from control-as-inference for single agents framework to improve the currents multi-agent algorithms, for example, enable it to do hierarchical controls to represent belief over states, or to develop an algorithm based on information theoretic views of control. This would proof the flexibility of our developed views. 
%     % \item There still exists some algorithms that are not part of MEMARL frameworks, thus, we will try to provided the best mapping from these algorithms to our frameworks. \item Finally, we will investigate the effects on the behavior the algorithms derived from MEMARL framework, since there has been concern regarding the theoretical performance of current control-as-inference frameworks, especially, exploration. After understanding the drawn backs from control-as-inference frameworks, we will proposed a solutions that address the problems posted, and provide both theoretical guarantee and empirical results. \textbf{(MAYBE)}
% \end{enumerate}
% The main aim of this thesis is to provide tools necessary to systematically derived multi-agent reinforcement learning based on Baysian inference, and to resolve inconsistency within the literature, mainly the derivation of the algorithms. Furthermore, all of the algorithms presented are examined theoretically and empirically against current state-of-the-art.

% \Phu{I am aimed for 1-3 and some algorithm within 4. However, if there is a time there is a chance of finishing all of them + this would be a good list of publications}

In this thesis, we aim to gain more understanding of family called maximum entropy multi-agent reinforcement learning (MEMARL), which is based on well-known single agent reinforcement learning framework called maximum entropy reinforcement learning (MERL) or control as inference\footnote{We will use these terms interchangeably}. that utilized variational inference. By framing multi-agent problems as probabilistic inference, one can include concepts such as recursive reasoning \cite{wen2019probabilistic, wen2019multi} into a multi-agent reinforcement learning algorithms, while being decentralized training and decentralized execution. However, there are some drawbacks with this approach, notably, its inability to train agents that have a conflict of interests. In this work, we try to solve this drawback by investigating the algorithm thoroughly, which leads to valuable insight which is hidden in plain sight. Furthermore, we will also argue in support of MEMARL framework by showing that many complex forms of agent can be systematically derived/reinterpret using techniques in this framework. Ultimately, one might view this thesis as a \textit{cookbook} for deriving multi-agent reinforcement learning, which we hope to be useful in future research. Our contributions are listed as follows:
\begin{itemize}
    \item Examine the existing algorithms withing MEMARL framework, what make them successful, and what their flaws are. We will also consider another algorithm called Balancing-Q learning \cite{grau2018balancing}, which is closely related to this framework but lacks proper probabilistic interpretation. By re-deriving results, we can discover the flaws and thus able to mitigate some of the problems. 
    \item After fixing the problem of the framework, we shall move on to apply the framework to various multi-agent problems that have different characteristic, notably delayed communication and representing public belief. We will provide reinterpretations of some of the algorithms in the literature, proving the strength and versatility of the framework.
\end{itemize}
Finally, looking at the bigger picture, we would like to show one of the ways multi-agent reinforcement algorithms could be derived by providing tools and patterns that one can deploy. At the same time, examine/create a new algorithm. And, by using the framework that is based on a popular single-agent framework, we can apply some of the works done in single-agent literature into the multi-agent domain with relative ease. 
\end{abstract}

\tableofcontents


\begin{figure}[!t]
    \centering
    \begin{chart}%\grid
        \reqhalfcourse 35,60:{}{Chapter 3}{}
        \reqhalfcourse 45,50:{}{Chapter 4}{}
        \reqhalfcourse 65,40:{}{Chapter 5}{}
        % \newcommand{\background}{green!15}
        \reqhalfcoursec 15,50:{}{Chapter 6}{}{black!10}
        \reqhalfcourse 35,40:{}{Chapter 7}{}
        
        \reqhalfcourse 42,25:{}{Chapter 8}{}
        \reqhalfcoursec 28,25:{}{Chapter 9}{}{black!10}
        
        \prereq 35,60,15,50:
        \prereqc 35,60,28,25;50:
        \prereq 35,60,45,50:
        \prereq 45,50,35,40:
        \prereq 45,50,65,40:
        \prereq 45,50,42,25:
        \coreqc 65,40,42,25;-30:
        \coreq 15,50,28,25:
        \coreq 35,40,28,25:
        
        \begin{pgfonlayer}{courses}
            \draw[dashed] ([shift={(-1mm,-1mm)}]x28y25.south west) rectangle ([shift={(1mm,1mm)}]x42y25.north east);
            % \node [text,align=center] [below=higher] {Higher-Level Reasoning};
        \end{pgfonlayer}
    \end{chart}
    \caption{The chapter dependencies, where chapter 8 and 9 are similar in the concept and motivation, and the dashed line represents, recommend but not required reading (there are some references to chapters before). Chapter 6 and Chapter 9 are an extension to a cooperative game hence the same colour, while the others deal with the training aspect of possible general sum game.}
\end{figure}

\setcounter{page}{1}

\chapter{Introduction}
\label{chapter:intro}
\begin{miniabstract}
    In this chapter, we will go through, in more details, the main problems in this thesis, structure of this thesis and the contributions made. At the start, we will provide major drawn backs into the current algorithms in the literature. After this, we will show the current development control-as-inference in single agent domains, which can be extends to multi-agent domain, showing possible profits from resolving the stated problem. Finally, we will expand the whole layout of this thesis.
\end{miniabstract}


\chapter{Background}

\chapter{Unified View of Probabilistic Multi-agent Reinforcement Learning}

\chapter{Solving Inconsistencies}

\chapter{EM-Algorithm For Multi-Agent Reinforcement Learning}

\chapter{Hierarchy and Communication in Multi-Agent Reinforcement Learning}

\chapter{Imitation Learning and Context Aware Multi-Agent Reinforcement Learning}

\chapter{Generalized Recursive Reasoning: revisited}

\chapter{Probabilistic perspective on Public Belief MDP}

\chapter{Conclusion}

\appendix

% \bibliographystyle{apalike}
\bibliographystyle{plain}
\bibliography{bibilography}

\chapter{Other Proofs}
\section {Probabilistic Machine Learning}
\subsection{ELBO from KL-divergence}
\label{appendix-1:elbo-kl}
We will start with the initial equation that we get and then expanded from that 
\begin{equation*}
    \begin{aligned}
        -\int q_{\phi} (\theta) \log\left( \frac{q_{\phi} (\theta)}{P(\theta | X, Y)} \right) \dby \theta &= \int q_{\phi} (\theta) \log\left( \frac{P(\theta | X, Y)}{q_{\phi} (\theta)} \right) \dby \theta \\ 
        &= \int q_{\phi} (\theta) \log\left( \frac{P(Y | X, \theta) P(\theta)}{P(Y|X)q_{\phi} (\theta)} \right) \dby \theta \\
        &= \int q_{\phi} (\theta) \log \left( \frac{P(\theta)}{q_{\phi} (\theta)} \right) + q_{\phi} (\theta) \log P(Y | X, \theta) - q_{\phi} (\theta) \log P(Y| X) \dby \theta \\
        &= \int q_{\phi} (\theta) \log\left( P(\theta | X, Y) \right) \dby \theta - \int q_{\phi} (\theta) \log\left( \frac{q_{\phi} (\theta)}{P(\theta)} \right) \dby \theta  - \const \\
        &= \mathbb{E}_{q_{\phi} (\theta)} \left[ P(Y | X, \theta) \right] - \kl \left(q_{\phi} (\theta) \Big\| P(\theta) \right) - \const \\
    \end{aligned}
\end{equation*}


\subsection{ELBO from Jensen's inequality}
\label{appendix-1:elbo-jensen}
We will start with the log-likelihood 
\begin{equation}
    l(X, Y) = \log \left( \int P(Y,  \theta | X) \dby \theta \right)
\end{equation}
Now we will introduce the variational distribution $q_{\phi}(\theta)$ that we would like to estimate
\begin{equation*}
    \begin{aligned}
        l(X, Y) &= \log \left( \int P(Y,  \theta | X) \dby \theta \right) \\
        &= \log \left( \int P(Y,  \theta | X) \frac{q_{\phi}(\theta)}{q_{\phi}(\theta)} \dby \theta \right) \\
        &\ge \int  q_{\phi}(\theta) \log\left(\frac{P(Y,  \theta | X)}{q_{\phi}(\theta)}\right) \dby \theta
    \end{aligned}
\end{equation*}
Since log function is concave, we can use Jensen's inequality to derived the lower bounded on the log-likelihood.


\subsection{Gap between ELBO and log-likelihood}
\label{appendix-1:elbo-gap}
We will show that the sum between the ELBO and KL-divergence between $P(\theta | X, Y)$ and $q_{\phi}(\theta)$ is indeed equal to the log-likelihood.
\begin{equation*}
    \int q_{\phi}(\theta) \log\left(\frac{P(Y, \theta | X)}{q_{\phi}(\theta)}\right) \dby \theta + \int q_{\phi} (\theta) \log\left( \frac{q_{\phi} (\theta)}{P(\theta | X, Y)} \right) \dby \theta
\end{equation*}
Starting by expanding the integral and algebraic manipulation, we can see that this is equal to \begin{equation*}
    \begin{aligned}
        \int q_{\phi}(\theta) &\log\left(\frac{P(Y, \theta | X)}{q_{\phi}(\theta)}\right) +  q_{\phi} (\theta) \log\left( \frac{q_{\phi} (\theta)}{P(\theta | X, Y)} \right) \dby \theta   \\ 
        &= \int q_{\phi}(\theta) \log\left( \frac{P(Y, \theta | X)}{P(\theta | X, Y)} \right) \dby \theta = \int q_{\phi}(\theta) \log\left( P(Y | X) \right) \dby \theta \\
        &= \log\left( P(Y | X) \right) \int q_{\phi}(\theta) \dby \theta = \log\left( P(Y | X) \right) \cdot 1
    \end{aligned}
\end{equation*}

\section{Reinforcement Learning}

\subsection{Policy Improvement Theorem}
\label{appendix-2:policy-im}

We start with the definition of improved policy:
\begin{equation*}
    \pi'(a | s) = \begin{cases}
        1 &\text{ if } a = \argmax{a \in A} Q^\pi(s, a) \\
        0 &\text{ otherwise }
    \end{cases}
\end{equation*}
We have the following sequence of substitutions
\begin{equation*}
    \begin{aligned}
        V^{\pi}(s) &\le Q^{\pi}(s, \pi'(s)) \\
        &= r(s, \pi'(s)) + \gamma \mathbb{E}_{s' \sim \mathcal{T}(s' | s, a)}\brackb{V^\pi(s')}  \\
        &\le r(s, \pi'(s)) + \gamma \mathbb{E}_{s' \sim \mathcal{T}(s' | s, a)}\brackb{Q^{\pi}(s', \pi'(s'))}  \\
        &\vdots \\
        &\le r(s, \pi'(s)) + \gamma  r(s', \pi'(s')) + \gamma^2  r(s'', \pi'(s'')) + \cdots \\
        &= V^{\pi'}(s)
    \end{aligned}
\end{equation*}
Please note that this improved policy is deterministic, therefore, we can simply substitute the output into the action value function.

\subsection{Construct Matrices for Reinforcement Learning}
\label{appendix-2:metrics-rl}

\subsection{Expected Bellman operator is contraction mapping}
\label{appendix-2:expect-bellman-contract}

\subsection{Banach fixed-point theorem}
\label{appendix-2:fixed-point}

\subsection{Optimal value Bellman update operator is contraction mapping}
\label{appendix-2:optimal-val-bellman-contract}

\subsection{Optimal action Bellman update operator is contraction mapping}
\label{appendix-2:optimal-q-bellman-contract}

% \chapter{Reinforcement Learning}
% %  We can expand the equality in equation \ref{eqn:Q-and-V} for the value function as follows:
% \begin{equation}

% \end{equation}
% We call this equation expected Bellman equation along with the following expected Bellman operator $\contractop^{\pi} : \mathbb{R}^{|S|} \rightarrow \mathbb{R}^{|S|}$ defined as:
% \begin{equation}
%     \label{eqn:exp-bellman-operator}
%     \contractop^{\pi} V(s) = \mathbb{E}_{a \sim \pi(a | s)} \brackb{r(s, a) + \gamma \mathbb{E}_{s' \sim \contractop(s' | s, a)}\brackb{V(s')}} 
% \end{equation}
% Normally we have to find $V(s)$ such that $\contractop^\pi V(s) = V(s)$, which involves matrix inversion (see appendix \ref{appendix-2:metrics-rl} for more details). However, turn out this expected Bellman operator is an contraction mapping on $\infty$-norm i.e 
% \begin{equation*}
%     \| \contractop^\pi V_1(s) - \contractop^\pi V_2(s) \|_\infty \le \alpha \|V_1(s) - V_2(s)\|_\infty
% \end{equation*}
% For some $0 \le \alpha < 1$ (See appendix \ref{appendix-2:expect-bellman-contract} for proof). And by the following theorem (We will also present the proof of this theorem in appendix \ref{appendix-2:fixed-point}) 
% \begin{theorem}{(Banach fixed-point theorem)}
%     Given the complete (every Cauchy sequence converges to a point in that space) metric space $(X, d(\cdot, \cdot))$ and the contraction mapping $\contractop : X \rightarrow X$, the sequence $(\contractop^{(n)} x)^{\infty}_{n=1}$ converges to a fixed point $x^*$ where $\contractop x^* = x^*$, for all points $x \in X$. 
% \end{theorem}
% Repeatedly apply the Bellman operator will lead us to a solution of $V^{\pi}$.  In conclusion, Policy iteration is an algorithm that solves the MDP by involving the alternation between the following 2 steps
% \begin{enumerate}
%     \item \textbf{Policy Evaluation}: Starting with randomized value function and Repeatedly applying Bellman operator defined in equation \ref{eqn:exp-bellman-operator} until converge to get true value function $V^{\pi}(s)$.
%     \item \textbf{Policy Improvement}: Update the policy by choosing the best action from action value function (we can calculate this by the equation \ref{eqn:Q-and-V}) following the equation \ref{eqn:greedy-Q}
% \end{enumerate}
% Since the value function always increases in every state, this algorithm is guaranteed to reach the optimal value function and policy. 

% \subsubsection{Value Iteration}
% Now, it is quite computational ineffective to evaluates the policy every times to update the policy. Can we just update the value function once and then uses the policy improvement to come-up with the next value function. We will define optimal value Bellman update operator as:
% \begin{equation}
%     \label{eqn:optimal-bellman-operator}
%     \contractop^*_V V(s) = \max_{a \in A}\Big(r(s, a) + \gamma \mathbb{E}_{s' \sim \mathcal{T}(s' | s, a)}\brackb{V(s')} \Big)
% \end{equation}
% In this case, we update the the value function toward the action that yields highest value given only one update. It is clear that $V^*(s)$ is the fixed point of this operator. This is, indeed, a contraction mapping, thus repeatedly updating the value function by optimal value Bellman operator will give us optimal value function. The prove will be appendix \ref{appendix-2:optimal-val-bellman-contract}. Finally, given the optimal value function, one can calculate the optimal action value function, and optimal policy. 

% \subsubsection{Q Iteration}
% \label{sec:Q-iteration}

% Now, instead of using value function as in value iteration, it is always desirable for us to estimate optimal action value function $Q^*(s, a)$, simply because the agent can greedily choose the action $a$ that maximizes this optimal action value function, see equation \ref{eqn:greedy-Q}. One can define the optimal action value Bellman operator as 
% \begin{equation}
%     \contractop^*_Q Q(s, a) = r(s, a) + \gamma \mathbb{E}_{s' \sim \mathcal{T}(s' | s, a)}\brackb{\max_{a\in A} Q(s', a)} 
% \end{equation}
% Similarly, it is clear that the optimal action value function is fixed point of this operator, while we can proof (in appendix \ref{appendix-2:optimal-q-bellman-contract}) that this operator is indeed a contraction mapping.

% \subsubsection{SARSA}
% All the algorithms presented till now are all model based algorithm, whereby we need a transition function $\mathcal{T}(s'|s, a)$ in order to solve MDP. This sometimes being intractable or unknown. From now on for the rest of the thesis, we will only consider the algorithm that doesn't require the knowledge of the environment, known as model-free. In model-free algorithm, we aim to estimate $Q^*(s, a)$ instead. That is because choosing action will be very easy if we get hold of $Q^*(s, a)$ simply choosing the action that maximizes the function given the state. Analogous to model based algorithm above, we want to update our current knowledge of $Q(s, a)$ and then do policy improvement on it following policy iteration, which we will call this SARSA. Before we move on, we would like to consider the general update rule at iteration $k$, which is 
% \begin{equation}
%     \label{eqn:train-q-val}
%     Q^{(k)}(s, a) \leftarrow Q^{(k)}(s, a) + \alpha\brackb{\text{target value} - Q^{(k)}(s, a)}
% \end{equation}
% from now on, we will only consider about the target value. In addition, the experience collected will be of the form $(s_{t}, a_{t}, r_{t}, s'_{t}, a'_{t})$. SARSA depends on the concept called bootstrap, where by after we collect the experience, the target action value at iteration $k$ is
% \begin{equation}
%     r(s_t, a_t) + \gamma Q^{(k)}(s_{t+1}, a_{t+1})
% \end{equation}
% The follows the connection between value function and action value function, but ignoring the effect of expectation on policy and transition function. This will give an biased estimate of the action value function, although being low in variance. The bias estimate can be mitigated by including samples in $N$ later times, for instance 
% \begin{equation}
%     q^{(N+1)}_k = r(s_t, a_t) + \sum^N_{n=1} \gamma^n r(s_{t+n}, a_{t+n}) + \gamma^{N+1} Q^{(k)}(s_{t+N+1}, a_{t+N+1})
% \end{equation}
% With this, we can even do weighted sum on the N-steps estimation given hyperparameter $\lambda$ as follows
% \begin{equation}
%     (1-\lambda)\sum^\infty_{n=1} \lambda^{n-1} q^{(n)}_k
% \end{equation}
% which we will call forward view of SARSA($\lambda$). For an online implementation and other variances (eligibility traces), we refer to chapter 12 of \cite{sutton2018reinforcement}. 

% \subsubsection{Q Learning}

% For Q-learning, we simply aim to approximate optimal action value function. This can be done off-policy (meaning that any experience tuple sample from any policy will be sufficient). Similar to SARSA, which is based on policy evaluation, we based the algorithm on Q Iteration (Section \ref{sec:Q-iteration}) in which we can update the current estimate by setting the target value as:
% \begin{equation}
%     r(s_t, a_t) + \gamma \max_{a'} Q^{(k)}(s_{t+1}, a')
% \end{equation}

% \Phu{There is Michael Jordan's Paper on the convergence of this + Study the convergence of these stuff}

\end{document}
